\chapter{Optics and apparent effects: special relativity applied to
astronomy}
\label{chap:astronomy}
\markboth{\MakeUppercase{Chapter~\thechapter.~~Optics and apparent effects}}{}

Up to now, we have always stipulated that observers making
measurements are endowed with divine knowledge and excellent data
analysis skills (recall Section~\ref{sec:observing}).  For example, in
Chapter~\ref{chap:time}, when E measured the rate of D's clock, she
did not simply measure the time between light pulses she received, she
corrected them for their light-travel times in getting from D's clock
to her eyes.  The corrections E made to the arrival times were only
possible because E was informed of D's trajectory before the
experiment.  Unfortunately, in many experiments, we do not know in
advance the trajectories of the objects we are studying.  This is
especially true in astronomy, a subject which, among other things,
attempts to reconstruct a 3+1-dimensional history of the Universe from
a set of 2-dimensional telescope pictures which span a very brief
duration in time (in comparison with the age of the Galaxy or
Universe).

In this chapter we discuss the appearance of objects to real
observers.


\section{Doppler shift (revisited)}
\label{sec:redshift}

Consider an object moving with respect to the Earth and which we are
observing from Earth.  Without loss of generality, we can choose the
coordinate system for the Earth's rest frame that puts the Earth at
the spatial origin, the moving object a distance $D$ away on the
positive $x$-axis, and puts the object's trajectory in the $x$-$y$
plane.  Its velocity vector $\tv{v}$ makes an angle $\theta$ with the
line of sight, as shown in Figure~\ref{fig:redshift}.
\begin{figure}
\horfigure{redshift.eps}
\caption[Object moving quickly with respect to the Earth]{An object
moving at relativistic velocity $\tv{v}$ with respect to the Earth
(symbolized by ``$\oplus$'') at an angle $\theta$ to the line of
sight.  Note that this is a diagram of space rather than spacetime.}
\label{fig:redshift}
\end{figure}

Let us imagine that the moving object emits pulses of light at
intervals of proper time $\Delta\tau_e$, and that its distance from us
$D$ is much greater than $c\,\Delta\tau_e$.  The question
we want to answer is this: If the light pulse emitted at time $0$
arrives at Earth at time $t=D/c$, how much later does the next pulse
arrive?

The next pulse is emitted a time $\Delta t_e=\gamma\,\Delta\tau_e$,
later (where $\gamma\equiv(1-\beta^2)^{-1/2}$, $\beta\equiv v/c$, and
$v\equiv |\tv{v}|$), at which time the object is $\Delta x=v\,\Delta
t_e\,\cos\theta$ further away, so the flash takes additional time
$\Delta x/c$ to get to us.  The time interval $\Delta t_r$ between
reception of the flashes is therefore
\begin{eqnarray}
\Delta t_r & = & \Delta t_e + \frac{v}{c}\,\Delta t_e\,\cos\theta \nonumber\\
& = & (1 + \beta\,\cos\theta)\,\gamma\,\Delta\tau_e \; .
\end{eqnarray}
If the motion is basically away from the Earth ($\theta<\pi$), the
time interval $\Delta t_r$ is longer than $\Delta\tau_e$.  The
analysis still holds if we take the two events not to be flashes, but
successive crests of an electromagnetic wave coming from the object.
The observed period is longer than the rest-frame period; the
observed frequency is lower than the rest-frame frequency; the
light is shifted to the red.

It is customary in astronomy to define a dimensionless {\em
redshift\/} $z$ by
\begin{eqnarray}
(1+z) & \equiv & \frac{\Delta t_r}{\Delta\tau_e} \nonumber\\
& = & \gamma\,(1+\beta\,\cos\theta) \; .
\end{eqnarray}
In the simple case $\theta=0$ (radial motion) the redshift is given by
\begin{equation}
(1+z) = \gamma\,(1+\beta) = \sqrt{\frac{1+\beta}{1-\beta}} \; ,
\end{equation}
and when $\theta=\pi$ (inward radial motion) the redshift $z$ is
{\em negative,\/} we call it a {\em blueshift\/} and it is given by
\begin{equation}
(1+z) = \gamma\,(1-\beta) = \sqrt{\frac{1-\beta}{1+\beta}} \; .
\end{equation}

Even when the motion is perfectly tangential, $\theta=\pi/2$, there is
a redshift which originates solely in the $\gamma$ factor.  This is
known as the {\em second-order\/} redshift and it has been observed in
extremely precise timing of high-velocity pulsars in the Galaxy.  Of
course all of these redshift effects are observed and have to be
corrected-for in tracking and communication between artificial
satellites.

Interestingly, the Doppler shift computed here, for the ratio of time
intervals between photon arrivals in two different frames, is just the
reciprocal of the Doppler shift formula computed in
Section~\ref{sec:vadd2}, for the ratio of photon energies in two
different frames.  In quantum mechanics, the energy of a photon is
proportional to the frequency of light, which is the reciprocal of the
time interval between arrivals of successive wave crests.  Quantum
mechanics and special relativity would be inconsistent if we did not
find the same formula for these two ratios.  Does this mean that
special relativity requires that a photon's energy be proportional to
its frequency?

\begin{problem}
The {\rm [O\,{\sc ii}]} emission line with rest-frame wavelength
$\lambda_0=3727\,\mbox{\AA}$ is observed in a distant galaxy to be at
$\lambda=9500\,\mbox{\AA}$.  What is the redshift $z$ and recession
speed $\beta$ of the galaxy?
\end{problem}

Light travels at speed $c$, so the observed wavelength $\lambda$ is
related to the observed period $T$ by $c\,T=\lambda$.  The rest-frame
wavelength $\lambda_0$ is related to the rest frame period $\tau$ by
$c\,\tau=\lambda_0$.  So
\begin{equation}
(1+z) \equiv \frac{T}{\tau} = \frac{\lambda}{\lambda_0}
 = \frac{9500\,\mbox{\AA}}{3727\,\mbox{\AA}} \; ;
\end{equation}
$z = 1.55$.  Assuming the velocity is radial,
\begin{eqnarray}
(1+z) & = & \sqrt{\frac{1+\beta}{1-\beta}} \nonumber\\
(1+z)^2-\beta\, (1+z)^2 & = & 1+\beta \nonumber\\
\beta & = & \frac{(1+z)^2-1}{(1+z)^2+1} \; ,
\end{eqnarray}
in this case we get $\beta=0.73$.  The galaxy is receding from us at
$0.73c$.


\section{Stellar Aberration}
\label{sec:aberration}

Imagine two observers, Ursula (U) and Virginia (V), both at the same
place, observing the same star, at the same time, but with V moving in
the $x$-direction at speed $v$ relative to U.  In U's frame, the star
is a distance $r$ away and at an elevation angle $\theta$ with respect
to the $x$-axis.  Light travels at speed $c$, so for any photon coming
from the star, the 4-displacement $\Delta\fv{x}$ between the event of
emission $E$ and observation $O$ in U's frame is
\begin{eqnarray}
\Delta\fv{x} & = & (c\Delta t,\Delta x,\Delta y,\Delta z)\nonumber\\
& = & (-r,r\,\cos\theta ,r\,\sin\theta ,0)
\end{eqnarray}
where the time component is negative because emission happens before
observation.  We apply the Lorentz transformation to get the
components in V's frame
\begin{eqnarray}
\Delta\fv{x} & = & (c\Delta t',\Delta x',\Delta y',\Delta z')\nonumber\\
& = & (-\gamma\,r-\gamma\,\beta\,r\,\cos\theta ,
\gamma\,r\,\cos\theta+\gamma\,\beta\,r ,r\,\sin\theta ,0)\nonumber\\
\end{eqnarray}
where, as usual, $\beta\equiv v/c$ and
$\gamma\equiv(1-v^2/c^2)^{-1/2}$.  Since the photons also travel at
speed $c$ in V's frame, we can re-write this in terms of the distance
$r'$ to the star and elevation angle $\theta'$ in V's frame:
\begin{equation}
\Delta\fv{x}=(-r',r'\,\cos\theta' ,r'\,\sin\theta' ,0)
\end{equation}
Solving for $\theta'$,
\begin{equation}
\cos\theta'=\frac{\cos\theta +\beta}{1+\beta\,\cos\theta}
\end{equation}
i.e., V observes the star to be at a different angular position than
that at which U does, and the new position does not depend on the
distance to the star.

This effect is {\em stellar aberration\/} and it causes the positions
on the sky of celestial bodies to change as the Earth orbits the
Sun.\footnote{Do not confuse this effect with parallax, which also
causes the positions to change, but in a manner which depends on
distance.}  The Earth's orbital velocity is $\sim30~{\rm km\,s^{-1}}$
($\beta=10^{-4}$), so the displacement of an object along a line of
sight perpendicular to the plane of the orbit (i.e., $\cos\theta=0$)
is on the order of $10^{-4}$ radians or $\sim 20$~arcseconds, a small
angle even in today's telescopes.  Despite this, the effect was first
observed in a beautiful experiment by Bradley in 1729.\footnote{The
paper is Bradley (1729); an excellent description and history of the
experiment is Shankland (1964).}

Notice that as the speed $v$ is increased, the stars are displaced
further and further towards the direction of motion.  If U is inside a
uniform cloud of stars and at rest with respect to them, V will see a
non-uniform distribution, with a higher density of stars in the
direction of her motion relative to the star cloud and a lower density
in the opposite direction.


\section{Superluminal motion}
\label{sec:superluminal}

It is observed that two components of the radio galaxy 3C~273 are
moving apart at $\mu=0.8$~milliarcseconds per year (Pearson et al
1981; recall that a milliarcsecond is $1/1000$ of $1/3600$ of a
degree).  From the known rate of expansion of the Universe and the
redshift of the radio galaxy, its distance\footnote{In cosmology,
there are many different ways of defining the ``distance'' betweeen
two objects, reviewed by Weinberg (1972, Chapter~14).  The ``proper
motion distance'' is used in this context.} $D$ from the Milky Way
(our own galaxy) has been determined to be $2.6\times 10^{9}$~light
years (a light year is the distance light travels in one year).  If we
multiply $\mu$ by $D$ we get the tangential component of the relative
velocity of the two components.  Because there can also be a radial
component, the velocity component we derive will be a lower limit on
the speed of the object.  Converting to radians we find $\mu=4\times
10^{-9}$~radians per year, so the tangential component of the velocity
is roughly $10$~light years per year!  This is faster than twice the
speed of light, the maximum relative speed at which we should ever
observe two objects to move.  Relative speeds exceeding $2c$ have now
been observed in many radio galaxies, and recently even in a jet of
material flowing out of a star in our own galaxy (Hjellming \& Rupen
1995); the effect has been dubbed {\em superluminal motion.\/} Is
relativity wrong and can things really exceed the speed of light?

Figure~\ref{fig:slm} depicts an object moving at a relativistic speed
$v=|\tv{v}|$ at an angle $\theta$ to the line of sight.  The object is
nearly moving directly towards the Earth, so $\theta$ is close to
$\pi$~radians or $180\degrees$.  The object emits flashes at events
$A$ and $B$, which are separated in time by $\Delta t_e$ in the
Earth's rest frame.  The distance between the events is much smaller
than the distance $D$ of the object from the Earth.
\begin{figure}
\horfigure{slm.eps}
\caption[Object moving quickly, nearly towards the Earth.]{An object
moving at relativistic velocity $\tv{v}$ on a trajectory that is
nearly straight towards the Earth.  The object emits flashes at points
$A$ and $B$.}
\label{fig:slm}
\end{figure}

What is the time interval $\Delta t_r$ between the receptions of the
two flashes at the Earth?  Flash $A$ takes time $D/c$ to get to us,
but flash $B$ takes only $D/c+(v\,\Delta t_e\,\cos\theta)/c$ to get to
us because the object is closer (note that $\cos\theta$ is negative).
So
\begin{equation}
\Delta t_r = \Delta t_e + \beta\,\Delta t_e\,\cos\theta \; ,
\end{equation}
where $\beta\equiv v/c$.  The tangential separation of events $A$ and
$B$ as seen from the Earth is $\Delta y = v\,\Delta t_e\,\sin\theta$,
so the inferred tangential velocity component is
\begin{equation}
v_{\rm inferred} = \frac{\Delta y}{\Delta t_r}
= \frac{\beta\,\sin\theta}{1+\beta\,\cos\theta}\,c \; ,
\end{equation}
which can be much bigger than $c$ if $\beta\approx 1$ and
$\cos\theta\approx -1$.

(It is worthy of note that there are many other possible explanations
for observed superluminal motions.  If the radio galaxy contains a
huge ``searchlight'' that sweeps its beam across intergalactic
material, the speed of the patch of illumination can certainly exceed
the speed of light.  Galaxies can act as gravitational ``lenses''
which distort and magnify background objects; this magnification can
make slowly-moving objects appear superluminal.  The moving patches
could be foreground objects, although this now appears {\em very\/}
unlikely.)

\begin{problem}
What is the minimum possible value of $\beta$ that could account
for the observed proper motion in 3C~273?  Assume that one component is
not moving tangentially with respect to the Earth and the other is.
\end{problem}


\section{Relativistic beaming}
\label{sec:beaming}

Consider an object emitting photons in all directions isotropically.
The {\em brightness\/} of the object is proportional to the amount of
radiation (energy per unit time) which the object emits into the pupil
of the observer's eye or telescope, and inversely proportional to the
solid angle (angular area, measured in square arcseconds, square
degrees, or steradians) occupied by the object.  The dimensions of
brightness are energy per unit time per unit solid angle.  Thus if two
objects emit the same amount of light, the more compact one is
brighter.  Brightness is a useful quantity in astronomy because it is
independent of distance: as a lightbulb is moved away from an
observer, the amount of light from the bulb entering the observer's
eye or telescope goes down as the inverse square of the distance, but
the solid angular size of the bulb also goes down as the inverse
square of the distance.  The brightness is constant.

Okay, the brightness of an object is independent of distance, but how
does it depend on how the object is {\em moving\/} relative to the
observer?  Doppler shift (Sections~\ref{sec:vadd2}
and~\ref{sec:redshift}) affects both the energy $E$ (or momentum $Q$)
of the photons and the rate of production $\Gamma$ of the photons
(i.e., number of photons emitted per unit time).  In addition, the
photon {\em directions\/} are different for the observer than for
someone in the rest frame of the object (as in stellar aberation,
Section~\ref{sec:aberration}), so the fraction of emitted photons
entering the observer's eye or telescope will also be affected by the
object's speed and direction.  For the same reason that in stellar
aberration (Section~\ref{sec:aberration}) observed star positions are
shifted into the direction of motion of the observer, emitted photons
are ``beamed'' into the direction of motion of the emitter.

Say the emitting object is at rest in frame $\rfr{S}'$, the rest
frame, but moving at speed $v=\beta c$ in the positive $x$-direction
in frame $\rfr{S}$, the frame of the observer.  In its rest frame, the
object emits photons of energy $E'=Q'c$ at rate $\Gamma'$ (photons per
unit time).  A photon emitted in a direction $\theta'$ relative to the
$x$-axis in frame $\rfr{S}'$ has 4-momentum
\begin{equation}
\fv{p}=(Q',Q'\,\cos\theta',Q'\,\sin\theta',0)
\end{equation}
where the $y$-direction has been chosen to make $p_z=0$
(Section~\ref{sec:compton}).  In frame $\rfr{S}$ it will have some
different momentum $Q$ and angle $\theta$ and the 4-momentum will be
\begin{equation}
\fv{p}=(Q,Q\,\cos\theta,Q\,\sin\theta,0)
\end{equation}
but it must be related to the 4-momentum in $\rfr{S}'$ by the Lorentz
transformation, so
\begin{eqnarray}
Q' & = & \gamma\,Q-\gamma\,\beta\,Q\,\cos\theta \nonumber \\
Q'\,\cos\theta' & = & \gamma\,Q\,\cos\theta-\gamma\,\beta\,Q
\end{eqnarray}
The first equation is just the Doppler shift (Sections~\ref{sec:vadd}
and~\ref{sec:redshift}); the ratio gives
\begin{equation}
\label{eqn:aber}
\cos\theta' = \frac{cos\theta-\beta}{1-\beta\,\cos\theta}
\end{equation}
which is exactly the same as the stellar aberration equation
(Section~\ref{sec:aberration}).

In the rest frame $\rfr{S}'$ the object emits isotropically, so the
rate per unit solid angle $\Omega$ (measured in steradians, or
radians$^2$) is just
\begin{equation}
\frac{d\Gamma'}{d\Omega'}=\frac{\Gamma'}{4\pi}
\end{equation}
which is independent of $\theta$.  In the observer frame $\rfr{S}$,
however, this will no longer be true.  Consider the solid-angular ring
of angular width $d\theta$ at angle $\theta$.  This ring has solid
angle
\begin{equation}
d\Omega = \sin\theta\, d\theta
\end{equation}
but the photons emitted into that ring in $\rfr{S}$ are emitted into a
different ring in $\rfr{S}'$ with solid angle
\begin{equation}
d\Omega' = \sin\theta'\, d\theta'
\end{equation}
where $\theta$ and $\theta'$ are related by (\ref{eqn:aber}).  Taking
the derivative of (\ref{eqn:aber})
\begin{eqnarray}
\sin\theta'\,d\theta' & = & \frac{\sin\theta\,d\theta}{1-\beta\,\cos\theta}
  + \frac{(\cos\theta-\beta)(\beta\,\sin\theta\,d\theta)}
         {(1-\beta\,\cos\theta)^2} \nonumber \\
 & = & \sin\theta\,d\theta \left[\frac{1-\beta^2}
         {(1-\beta\,\cos\theta)^2}\right] \nonumber \\
 & = & \frac{\sin\theta\,d\theta}{\gamma^2\,(1-\beta\,\cos\theta)^2}
\end{eqnarray}
so the ratio of solid angles is
\begin{equation}
\frac{d\Omega}{d\Omega'}=\gamma^2\,(1-\beta\,\cos\theta)^2
\end{equation}
the square root of which is the ratio of energies $E'/E$ (by the
Doppler shift) or the ratio of rates of photon production
$\Gamma'/\Gamma$ (by the same).  Putting it all together, since the
inferred brightness is proportional to the energies times the rate
divided by the solid angle, the ratio of brightness $I/I'$ between the
observer and rest frames is
\begin{equation}
\frac{I}{I'} = \left[\gamma\,(1-\beta\,\cos\theta)\right]^{-4}
\end{equation}
or in terms of redshift, $(1+z)^{-4}$!

\begin{problem}
Plot the observed brightness $I$ as a function of angle $\theta$
according to an observer at rest in $\rfr{S}$ observing an object
radiating isotropically in its rest frame $\rfr{S}'$.
\end{problem}


\section{The appearance of passing objects}
\label{sec:appearance}

Consider a rectangular plank of rest dimensions\footnote{The ``rest
dimensions'' are the dimensions the object has in its rest frame.}
$X\times Y$ moving at speed $v=\beta c$ in the $x$-direction,
perpendicular to the line of sight to a distant observer, as shown in
Figure~\ref{fig:plank1}.
\begin{figure}
\horfigure{plank1.eps}
\caption[A moving plank]{
A plank of rest dimensions $X\times Y$ moves at speed $\beta$
perpendicular to the line of sight to a distant observer.}
\label{fig:plank1}
\end{figure}
The light coming from the corners marked A and B get to the observer
before the light coming from corner C by a time interval $Y/c$.  For
this reason, at any instant of time, the plank will appear ``rotated''
to the observer, as you will show in the problems.  There is a nice
discussion of this apparent rotation effect in French (1966,
pp.~149--152).  The apparent rotation actually needs to be taken into
account by astrophysicists modeling features in relativistic jets
emitted by radio galaxies and stars (e.g., Lind \& Blandford 1985).

\begin{problem}
What is the apparent position of corner C to the observer in
Figure~\ref{fig:plank1} at the time that the light from corners A and
B reach the observer?  From this information, as well as length
contraction, compute the apparent locations of all four corners.
\end{problem}

\begin{problem}
Why doesn't the observer see corner D?
\end{problem}


\section{A simpleminded cosmology}
\label{sec:cosmology}

We know that the Universe is expanding.  In fact, we know that except
for a few, very close neighbours, other galaxies are receding from our
own and recession speed is proportional to distance from us.  This
effect is known as the {\em Hubble flow,\/} named after the astronomer
who first discovered it (Hubble, 1929).  This Hubble flow is naturally
explained by a simple cosmological scenario in which the Universe
begins with an explosion, and this scenario does not require our
galaxy to be at the center.

Consider an infinite Lorentz frame $\rfr{S}'$ with a small rock at
rest at the origin.  At time $t'=0$ in this frame, the rock explodes
into countless tiny fragments with masses small enough to ensure that
gravitational forces do not significantly affect the constant-velocity
(speed and direction) trajectories.  At time $t'>0$, there is some
distribution of fragments in space, with the faster-moving fragments
further out from the explosion point.  Because all fragment world
lines are constant-velocity and pass through the event (0,0,0,0) in
$\rfr{S}'$, the vector displacement $\tv{r}'$ in frame $\rfr{S}'$ of
a fragment with velocity $\tv{v}'$ is given by $\tv{r}'=\tv{v}'\,
t'$.

Now consider another frame $\rfr{S}$ which also has the explosion at
the origin, but which is moving along with one of the fragments not at
rest in $\rfr{S}'$.  In $\rfr{S}$, all the fragments have
constant-velocity worldlines that pass through the event (0,0,0,0).
Therefore in $\rfr{S}$ also, the displacement $\tv{r}$ at time $t>0$
of a fragment with velocity $\tv{v}$ is given by $\tv{r} =\tv{v}\, t$.
That is, at any time $t>0$, recession speed is proportional to
distance from the origin even though the origin is not at (or even at
rest with respect to) the center of the explosion.

If at time $t=t_0$ (now) we live on a fragment (the Milky Way) ejected
by a huge explosion (the Big Bang) which occurred at time $t=0$, and
the fragments are not heavy enough to have significantly affected each
other's velocities via gravitational forces, then by the above
argument we expect to see other nearby fragments (other galaxies)
receding from us, with their recession speeds proportional to their
distances from us; i.e.\ we expect a spherically symmetric Hubble
flow even if we are not at the center of the Universe.

Of course when we look at an extremely distant object now, we are not
seeing the object at its current position $\tv{r} (t_0)$, but rather at
its position $\tv{r} (t_e)$ at time $t_e$ when it emitted the light that
is now reaching us.  Also, we have no direct measure of the distance
$r_e=|\tv{r} (t_e)|$, but we can infer it from the redshift $z$ of the
light that it emits in, say, its hydrogen recombination lines (the
rest-frame frequencies of which we know).  What is the relationship
between $z$ and $r_e$?

\begin{figure}
\verfigure{cosmology.eps}
\caption[Cosmological spacetime diagram]{
The spacetime diagram used to derive the redshift-distance relation in
a simpleminded cosmology.  World lines of the Earth (vertical) and the
fragment (slope $1/\beta$) are shown.  Event $B$ is the big bang, $E$
the emission of light and $O$ its observation now on Earth.}
\label{fig:cosmology}
\end{figure}

Figure~\ref{fig:cosmology} is the spacetime diagram for a fragment
moving in $\rfr{S}$ (where $\rfr{S}$ is our rest frame) at velocity
$\tv{v}$, with coordinates aligned so that $\tv{v}$ points in the
$x$-direction.  The Big Bang occurs at event $B$, the origin
($c\,t=x=0$); the fragment emits light at event $E$ ($c\,t=c\,t_e$,
$x=r_e$); and we observe the light at event $O$, now ($c\,t=c\,t_0$,
$x=0$).

It should be obvious from the diagram that $c\,t_0=r_e/\beta+r_e$, where
$\beta=|\tv{v}|/c$ and that the proper time $\tau_{BE}$ elapsed for
the fragment between $B$ and $E$ is given by
$(c\,\tau_{BE})^2=(r_e/\beta)^2-r_e^2$.  From these relations and the
fact that the redshift $z$ is given by $1+z=t_0/\tau_{BE}$
(Section~\ref{sec:redshift}) it is easy to show that
\begin{equation}
\label{eqn:angdidis}
r_e=c\,t_0\,\frac{2z+z^2}{2(1+z)^2}
\end{equation}
(the student is encouraged to show this). It should be obvious, both
from Figure~\ref{fig:cosmology} and the above equation, that the
maximum value for $r_e$ is $c\,t_0/2$ when $z\rightarrow\infty$, and
that for small $z$, $r_e=c\,t_0\,z$.

In addition to inference from redshift, the distance to a fragment can
be determined several other ways.  If one knows the size of the
fragment, its angular diameter can be measured, and the ratio of the
quantities should provide the distance $r_e$.  For this reason, $r_e$
is referred to as the {\em angular diameter
distance}\footnote{Experienced cosmologists will notice that equation
(\protect\ref{eqn:angdidis}) is identical in form to the equation
derived, via general relativity, for the angular diameter distance in
a ``spatially curved, isotropic, homogeneous, empty space.''  See,
e.g., Weinberg (1972) or Peebles (1993) for the general-relativistic
derivation.} to the object, and is often denoted $d_{\rm A}$. If the
intrinsic luminosity $L$ of a fragment is known, its flux $F$ can be
measured, and the relation $F=L/(4\pi r^2)$ can be used to determine a
distance.  However, the {\em luminosity distance\/} $d_{\rm L}$
determined in this way is different from $d_{\rm A}$ by four factors
of $(1+z)$ because of the effect of redshift on brightness discussed
in Section~\ref{sec:beaming}.

The cosmology presented in this section is a simple Milne cosmology, a
more general version of which (including gravity) is described by
Milne (1934).  Most cosmologists now believe that the expansion of the
Universe is governed by general relativity, but it is nonetheless true
that most cosmological observations can be explained by this simple
kinematic model.
