\chapter{Relativistic mechanics}
\label{chap:mechanics}


\section{Scalars}
\label{sec:scalars}

A {\em scalar\/} is a quantity that is the same in all reference
frames, or for all observers.  It is an {\em invariant\/} number.  For
example, the interval $(\Delta s)^2$ separating two events $A$ and $B$
is a scalar because it is the same in all frames.  Similarly, the
proper time $\Delta\tau$ between two events on a worldline is a
scalar.  In Chapter~\ref{chap:time}, the number of ticks of D's clock
in going from planet A to planet B is a scalar because although
observers disagree on how far apart the ticks are in time, they agree
on the total number.

It is worth emphasizing that the time interval $\Delta t$ between two
events, or the distance $\Delta x$ between two events, or the the
length $\ell$ separating two worldlines are {\em not\/} scalars: they
do not have frame-independent values\footnote{Forget high
school---where all single-component numbers were probably referred to
as ``scalars.''}.


\section{4-vectors}
\label{sec:4-vectors}

Between any two distinct events $A$ and $B$ in spacetime, there is a
time difference $c\,\Delta t$ and three coordinate difference $\Delta
x$, $\Delta y$ and $\Delta z$.  These four numbers can be written as a
vector $\fv{x}$ with four components, which is called a {\em
4-vector\/}:
\begin{equation}
\fv{x} = (c\,\Delta t, \Delta x, \Delta y, \Delta z)
\end{equation}
The 4-vector\footnote{The convention in these notes is to denote
4-vectors with vector hats and 3-vectors with bold face symbols.}
$\fv{x}$ is actually a frame-independent object, although this is a
fairly subtle concept.  The components of $\fv{x}$ are not
frame-independent, because they transform by the Lorentz
transformation (Section~\ref{sec:lt}).  But event $A$ is frame
independent: if it occurs in one frame, it must occur in all frames,
and so is event $B$, so there is some frame-independent meaning to the
4-vector displacement or {\em 4-displacement\/} between these events:
it is the 3+1-dimensional arrow in spacetime that connects the two
events.

The frame-independence can be illustrated with an analogy with
3-dimensional space.  Different observers set up different coordinate
systems and assign different coordinates to two points $P$ and $Q$,
say Pittsburgh, PA and Queens, NY.  Although both observers agree that
they are talking about Pittsburgh and Queens, they assign different
coordinates to the points.  The observers can also discuss the
3-displacement $\tv{r}$ separating $P$ and $Q$.  Again, they may
disagree on the coordinate values of this 3-vector, but they will
agree that it is equal to the vector that separates points $P$ and
$Q$.  They will also agree on the length of $\tv{r}$ and they will
agree on the angle it makes with any other vector $\tv{s}$, say the
vector displacement between $P$ and $R$ (Richmond, VA).  In this sense
the points and 3-vector are frame-independent or coordinate-free
objects, and it is in the same sense that events and 4-vectors are
frame-independent objects.

With each 4-displacement we can associate a scalar: the interval
$(\Delta s)^2$ along the vector.  The interval associated with
$\fv{x}$ is
\begin{equation}
(\Delta s)^2 = (c\,\Delta t)^2 - (\Delta x)^2 - (\Delta y)^2 - (\Delta
z)^2
\end{equation}
Because of the similarity of this expression to that of the dot
product between 3-vectors in three dimensions, we also denote this
interval by a dot product and also by $|\fv{x}|^2$:
\begin{equation}
\fv{x}\cdot\fv{x} \equiv \left|\fv{x}\right|^2 \equiv
(c\,\Delta t)^2 - (\Delta x)^2 - (\Delta y)^2 - (\Delta z)^2
\end{equation}
and we will sometimes refer to this as the {\em magnitude\/} or {\em
length\/} of the 4-vector.

We can generalize this dot product to a dot product between any two
4-vectors $\fv{a}=(a_t,a_x,a_y,a_z)$ and $\fv{b}=(b_t,b_x,b_y,b_z)$:
\begin{equation}
\fv{a}\cdot\fv{b} \equiv
a_t\, b_t - a_x\, b_x - a_y\, b_y - a_z\, b_z
\end{equation}
It is easy to show that this dot product obeys the rules we expect dot
products to obey: associativity over addition and commutativity.  The
nice result is that the dot product produces a scalar.  That is, the
dot product of any two 4-vectors in one frame equals their dot product
in any other frame.

When frames are changed, 4-displacements transform according to the
Lorentz transformation.  Because 4-displacements are 4-vectors, it
follows that all 4-vectors transform according to the Lorentz
transformation.  This provides a simple (though slightly out-of-date)
definition of a 4-vector: an ordered quadruple of numbers that
transforms according to the Lorentz transformation.

Because scalars, by definition, do not change under a Lorentz
transformation, any 4-component object which transforms according to
the Lorentz transformation can be multiplied or divided by a scalar to
give a new four-component object which also transforms according to
the Lorentz transformation.  In other words, a 4-vector multiplied or
divided by a scalar is another 4-vector.

\begin{problem}
Show that the 3+1-dimensional dot product obeys associativity
over addition, i.e., that
\begin{equation}
\fv{a}\cdot(\fv{b}+\fv{c}) = \fv{a}\cdot\fv{b}+\fv{a}\cdot\fv{c}
\end{equation}
and commutativity, i.e., that $\fv{a}\cdot\fv{b}=\fv{b}\cdot\fv{a}$.
\end{problem}

\begin{problem}
 Show that the dot product of two 4-vectors is a scalar.  That is,
show that for any two 4-vectors $\fv{a}$ and $\fv{b}$, their dot
product in one frame $\rfr{S}$ is equal to their dot product in
another $\rfr{S'}$ moving with respect to $\rfr{S}$.
\end{problem}

\begin{problem}
Show that 4-vectors are closed under addition.  That is, show
that for any two 4-vectors $\fv{a}$ and $\fv{b}$, their sum
$\fv{c}=\fv{a}+\fv{b}$ (i.e., each component of $\fv{c}$ is just the
sum of the corresponding components of $\fv{a}$ and $\fv{b}$) is also
a 4-vector.  Show this by comparing what you get by Lorentz
transforming and then summing with what you get by summing and then
Lorentz transforming.
\end{problem}


\section{4-velocity}
\label{sec:4-vel}

What is the 3+1-dimensional analog of velocity?  We want a 4-vector so
we want a four-component object that transforms according to the
Lorentz transformation.  In 3-dimensional space, 3-velocity $\tv{v}$
is defined by
\begin{equation}
\tv{v} \equiv \lim_{\Delta t\rightarrow 0} \frac{\Delta\tv{r}}{\Delta t}
= \frac{d\tv{r}}{dt}
\end{equation}
where $\Delta t$ is the time it takes the object in question to go the
3-displacement $\Delta\tv{r}$.  The naive 3+1-dimensional
generalization would be to put the 4-displacement $\Delta\fv{x}$ in
place of the 3-displacement $\Delta\tv{r}$.  However, this in itself
won't do, because we are dividing a 4-vector by a non-scalar (time
intervals are not scalars); the quotient will not transform according
to the Lorentz transformation.  The fix is to replace $\Delta t$ by
the proper time $\Delta\tau$ corresponding to the interval of the
4-displacement; the 4-velocity $\fv{u}$ is then
\begin{equation}
\fv{u}\equiv\lim_{\Delta\tau\rightarrow 0}\frac{\Delta\fv{x}}{\Delta\tau}
\end{equation}
When we take the limit we get derivatives, and the proper time
$\Delta\tau$ is related to the coordinate time $\Delta t$ by
$\gamma\,\Delta\tau =\Delta t$ (where, as usual, $\gamma\equiv
(1-\beta^2)^{-1/2}$ and $\beta\equiv |\tv{v}|/c$), so
\begin{eqnarray}
\fv{u} & = & \frac{d\fv{x}}{d\tau} \nonumber\\
& = & \left(c\,\frac{dt}{d\tau},\frac{dx}{d\tau},
            \frac{dy}{d\tau},\frac{dz}{d\tau} \right) \nonumber\\
& = & \left(c\,\gamma\,\frac{dt}{dt},\gamma\,\frac{dx}{dt},
            \gamma\,\frac{dy}{dt},\gamma\,\frac{dz}{dt} \right) \nonumber\\
& = & (\gamma\,c,\gamma\,v_x,\gamma\,v_y,\gamma\,v_z)
\end{eqnarray}
where $(v_x,v_y,v_z)$ are the components of the 3-velocity
$\tv{v}=d\tv{r}/dt$.  Although it is unpleasant to do so, we often
write 4-vectors as two-component objects with the first component a
single number and the second a 3-vector.  In this notation
\begin{equation}
\fv{u} = (\gamma\,c,\gamma\,\tv{v})
\end{equation}

What is the magnitude of $\fv{u}$?  There are several ways to derive
it, the most elegant is as follows.  The magnitude $|\fv{u}|^2$ must
be the same in all frames because $\fv{u}$ is a four-vector.  Let us
change into the frame in which the object in question is at rest.  In
this frame $\fv{u}=(c,0,0,0)$ because $\tv{v}=(0,0,0)$ and $\gamma=1$.
Clearly in this frame $|\fv{u}|^2=c^2$ or $|\fv{u}|=c$.  It is a
scalar so it must have this value in all frames.  Thus $|\fv{u}|=c$ in
all frames.  This trivial ``proof'' is a good model for
problem-solving in special relativity: identify something which is
frame-independent, transform into a frame in which it is easy to
calculate, and calculate it.  The answer will be good for all frames.

The reader may find this a little strange.  Some particles move
quickly, some slowly, but for all particles, the magnitude of the
4-velocity is $c$.  But this is not strange, because we need the
magnitude to be a scalar, the same in all frames.  If I change frames,
some of the particles that were moving quickly before now move slowly,
and some of them are stopped altogether.  Speeds (magnitudes of
3-velocities) are relative; the magnitude of the 4-velocity has to be
invariant.

\begin{problem}
Apply the formula for the magnitude of a 4-vector to the general
4-velocity $(\gamma\,c,\gamma\,v_x,\gamma\,v_y,\gamma\,v_z)$ to show
that its magnitude is indeed $c$.
\end{problem}


\section{4-momentum, rest mass and conservation laws}
\label{sec:4-mom}

Just as in non-relativistic 3-space, where 3-momentum was defined as
mass times 3-velocity, in spacetime 4-momentum $\fv{p}$ is mass $m$
times 4-velocity $\fv{u}$.  Under this definition, the mass must be a
scalar if the 4-momentum is going to be a 4-vector.  If you are old
enough, you may have heard of a quantity called ``relativistic mass''
which increases with velocity, approaching infinity as an object
approaches the speed of light.  Forget whatever you heard; that
formulation of special relativity is archaic and ugly.  The mass $m$
of an object as far as we are concerned is its {\em rest mass,\/} or
the mass we would measure if we were at rest with respect to the
object.

Rest mass is a scalar in that although different observers who are all
moving at different speeds with respect to the object may, depending
on the nature of their measuring apparati, measure different masses for
an object, they all can agree on what its mass would be if they were
at rest with respect to it.  In this respect rest mass is like the
proper time scalar: the only observers whose clocks actually measure
the proper time between two events are the observers for whom the two
events happen in the same place.  But all observers agree on what that
proper time is.

The 4-momentum $\fv{p}$ is thus
\begin{eqnarray}
\fv{p} & \equiv & m\,\fv{u} \nonumber\\
& = & (\gamma\,m\,c,\gamma\,m\,v_x,\gamma\,m\,v_y,\gamma\,m\,v_z) \nonumber\\
& = & (\gamma\,m\,c,\gamma\,m\,\tv{v})
\end{eqnarray}
Again, by switching into the rest frame of the particle, we find that
$|\fv{p}|=m\,c$.  This is also obvious because $\fv{p}=m\,\fv{u}$ and
$|\fv{u}|=c$.  As with 4-velocity, it is strange but true that the
magnitude of the 4-momentum does not depend on speed.  But of course
it cannot, because speeds are relative.

Why introduce all these 4-vectors, and in particular the 4-momentum?
In non-relativistic mechanics, 3-momentum is conserved.  However, by
Einstein's principle, all the laws of physics must be true in all
uniformly moving reference frames.  Because only scalars and 4-vectors
are truly frame-independent, relativistically invariant conservation
of momentum must take a slightly different form: in all interactions,
collisions and decays of objects, {\em the total 4-momentum is
conserved\/}.  Furthermore, its time component is energy $E/c$ (we
must divide by $c$ to give it the same dimensions as momentum) and its
spatial components make up a correct, relativistic expression for the
3-momentum $\tv{p}$.  We are actually re-defining $E$ and $\tv{p}$ to
be
\begin{eqnarray}
E & \equiv & \gamma\, m\, c^2 \nonumber\\
\tv{p} & \equiv & \gamma\, m\, \tv{v}
\end{eqnarray}
Please forget any other expressions you learned for $E$ or $\tv{p}$ in
non-relativistic mechanics.  Those other expressions are only good
when speeds are much smaller than the speed of light.

A very useful equation suggested by the new, correct expressions for
$E$ and $\tv{p}$ is
\begin{equation}
\tv{v} = \frac{\tv{p}\,c^2}{E}
\end{equation}

By taking the magnitude-squared of $\fv{p}$ we get a relation between
$m$, $E$ and $p\equiv |\tv{p}|$,
\begin{equation}
|\fv{p}|^2 = m^2\, c^2 = \left(\frac{E}{c}\right)^2 - p^2
\end{equation}
which, after multiplication by $c^2$ and rearrangement becomes
\begin{equation}
E^2 = m^2\, c^4 + p^2\, c^2
\end{equation}
This is the famous equation of Einstein's, which becomes $E=m\,c^2$
when the particle is at rest ($p=0$).\footnote{A friend of mine once
was passed by a youth-filled automobile, the contents of which
identified him as a physicist and shouted ``Hey nerd: $E=mc^2$!''
What has just been discussed explains why he ran down the street after
the automobile shouting ``Only in the rest frame!''}

If we take the low-speed limits, we should be able to
reconstruct the non-relativistic expressions for energy $E$ and
momentum $\tv{p}$.  In the low-speed limit $\beta\equiv v/c\ll 1$, and
we will make use of the fact that for small $\epsilon$,
$(1+\epsilon)^n\approx 1+n\,\epsilon$.  At low speed,
\begin{eqnarray}
\tv{p} & = & m\,\tv{v}\, (1-\beta^2)^{-1/2} \nonumber\\
& \approx & m\,v + \frac{1}{2}\,m\,\frac{v^2}{c^2}\,\tv{v} \nonumber\\
& \approx & m\,\tv{v} \nonumber\\
E & = & m\, c^2\, (1-\beta^2)^{-1/2} \nonumber\\
& \approx & m\,c^2 + \frac{1}{2}\,m\,v^2
\end{eqnarray}
i.e., the momentum has the classical form, and the energy is just
Einstein's famous $m\,c^2$ plus the classical kinetic energy
$m\,v^2/2$.  But remember, these formulae only apply when $v\ll c$.

Conservation of 4-momentum is just like conservation of 3-momentum in
non-relativistic mechanics.  All the 4-momenta of all the components
of the whole system under study are summed before the interaction, and
they are summed afterwards.  No matter what the interaction, as long
as the whole system has been taken into account (i.e.\ the system is
{\em isolated\/}), the total 4-momentum $\fv{p}$ before must equal the
total 4-momentum $\fv{q}$ after. In effect this single conservation
law $\fv{p}=\fv{q}$ summarizes four individual conservation laws, one
for each component of the 4-momentum.


\section{Collisions}
\label{sec:collisions}

It is now time to put conservation of 4-momentum into use by solving
some physics problems.  The essential technique is to sum up the total
4-momentum before and total 4-momentum after and set them equal.  But
just as in non-relativistic mechanics, there are tricks to learn and
there are easy and difficult ways of approaching each problem.

In non-relativistic mechanics, collisions divide into two classes:
elastic and inelastic.  In elastic collisions, both energy and
3-momentum are conserved.  In inelastic collisions, only 3-momentum is
conserved.  Energy is not conserved because some of the initial
kinetic energy of the bodies or particles gets lost to heat or
internal degrees of freedom.  In relativistic mechanics, 4-momentum,
and in particular the time component or energy, is conserved in {\em
all\/} collisions; no distinction is made between elastic and
inelastic collisions.  As we will see, this is because the {\em
correct,\/} relativistic expression we now use for energy takes all
these contributions into account.

In Figure~\ref{fig:collision}, a ball of putty of mass $m$ is
travelling at speed $v$ towards another ball of putty, also of mass
$m$, which is at rest.  They collide and stick forming a new object
with mass $M'$ travelling at speed $v'$.
\begin{figure}
\horfigure{collision.eps}
\caption[Collision diagram]
{(a)~A ball of putty of mass $m$ travels at speed $v$ towards an
identical ball which is at rest.  (b)~After the collision the balls
are stuck together and the combined lump has mass $M'$ and speed
$v'$.}
\label{fig:collision}
\end{figure}
In a non-relativistic world, $M'$ would be $2m$ and $v'$ would be
$v/2$, a solution that conserves non-relativistic momentum but not
non-relativistic energy; classically this collision is inelastic.  But
in a relativistic world we find that the non-relativistic predictions
for $v'$ and $M'$ are not correct and both energy and 3-momentum will
be conserved.

Before the collision, the 4-momentum of the moving ball is
$\fv{p}_m=(\gamma\, m\, c,\gamma\, m\, v,0,0)$, where I have aligned
the $x$-axis with the direction of motion, and of course
$\gamma\equiv(1-v^2/c^2)^{-1/2}$.  The 4-momentum of the stationary
ball is $\fv{p}_s=(m\, c,0,0,0)$, so the total 4-momentum of the
system is
\begin{equation}
\fv{p}=\fv{p}_m+\fv{p}_s=([\gamma+1]\,m\,c,\gamma\,m\,v,0,0)
\end{equation}
After the collision, the total 4-momentum is simply
\begin{equation}
\fv{q}=(\gamma'\,M'\,c,\gamma'\,M'\,v',0,0)
\end{equation}
where $\gamma'\equiv(1-v'^2/c^2)^{-1/2}$.

By conservation of 4-momentum, $\fv{q}=\fv{p}$, which means that the
two 4-vectors are equal, component by component, or
\begin{eqnarray}
\gamma'\,M'\,c & = & [\gamma+1]\,m\,c \nonumber\\
\gamma'\,M'\,v' & = & \gamma\,m\,v
\end{eqnarray}
The ratio of these two components should provide $v'/c$; we find
\begin{equation}
v' = \frac{\gamma\,v}{\gamma+1} > \frac{v}{2}
\end{equation}
The magnitude of $\fv{q}$ should be $M'\,c$; we find
\begin{eqnarray}
M'^2 & = & [\gamma+1]^2\,m^2 - \gamma^2\,m^2\,\frac{v^2}{c^2} \nonumber\\
& = & \left[1 + 2\,\gamma + \gamma^2\,\left(1-\frac{v^2}{c^2}\right)\right]\,m^2 \nonumber\\
& = & 2\,(\gamma+1)\,m^2 \nonumber\\
M' & = & \sqrt{2\,(\gamma+1)}\,m \nonumber\\
& > & 2\, m
\end{eqnarray}
So the non-relativistic answers are incorrect, and most disturbingly,
the mass $M'$ of the final product is greater than the sum of the
masses of its progenitors, $2\,m$.

Where does the extra rest mass come from?  The answer is energy.  The
collision is classically inelastic.  This means that some of the
kinetic energy is lost.  But energy is conserved, so the energy is not
actually lost, it is just converted into other forms, like heat in the
putty, or rotational energy of the combined clump of putty, or in
vibrational waves or sound traveling through the putty.  Strange as it
may sound, this internal energy actually {\em increases the mass\/} of
the product of the collision.

The consequences of this are strange.  For example, a brick becomes
more massive when one heats it up.  Or, a tourist becomes less massive
as he or she burns calories climbing the steps of the Eiffel
Tower.\footnote{Relativity does {\em not\/} provide the principal
reason that one can lose weight by excercising; you do the math.}  Or,
a spinning football hits a football player with more force than a
non-spinning one.  All these statements are true, but it is important
to remember that the effect is very very small unless the internal
energy of the object in question is on the same order as $m\,c^2$.
For a brick of $1~{\rm kg}$, that energy is $10^{20}$~Joules, or
$3\times10^{13}$~kWh, or my household energy consumption over about
ten billion years (roughly the age of the Universe).  For this reason,
macroscopic objects (like bricks or balls of putty) cannot possibly be
put into states of relativistic motion in Earth-bound experiments.
Only subatomic and atomic particles can be accelerated to relativistic
speeds, and even these require huge machines (accelerators) with huge
power supplies.

\begin{problem}
Suppose the two balls of putty in Figure~\ref{fig:collision} do not
hit exactly head-on but rather at a slight perpendicular displacement,
so in the final state the combined lump is spinning?  How will this
affect the final speed $v'$?  And the final mass $M'$?  Imagine now
that you stop the combined lump from spinning---will its mass be
greater than, equal to, or less than $M'$?
\end{problem}


\section{Photons and Compton scattering}
\label{sec:compton}

Can something have zero rest mass?  If we blindly substitute $m=0$
into Einstein's equation $E^2=m^2\,c^4+p^2\,c^2$ we find that $E=p\,c$
for a particle with zero rest mass (here $p$ is the magnitude of the
3-momentum).  But $v=p\,c^2/E$, so such massless particles would
always have to travel at $v=c$, the speed of light.  Strange.

Of course photons, or particles of light, have zero rest mass, and
this is ``why'' they always travel at the speed of light.  The
magnitude of a photon's 4-momentum is zero, but this does not mean
that the components are all zero; it just means that when the
magnitude is calculated, the time component squared, $E^2/c^2$, is
exactly cancelled out by the sum of the space components squared,
$p_x^2+p_y^2+p_z^2=|\tv{p}|^2$.  Thus the photon may be massless, but
it carries momentum and energy, and it should obey the law of
conservation of 4-momentum.  This was beautifully predicted and tested
in the famous Compton scattering experiment.  We outline the theory
behind this experiment here.

Figure~\ref{fig:compton} shows the schematic for Compton scattering.
A photon of initial 3-momentum magnitude $Q$ (or energy $Qc$)
approaches an electron of mass $m$ that is essentially at rest.  The
photon scatters off of the electron, leaving at some angle $\theta$ to
the original direction of motion, and with some new momentum $Q'$ (or
energy $Q'c$).  The electron leaves at some other angle $\phi$ and
some speed $v$.
\begin{figure}
\horfigure{compton.eps}
\caption{Before and after pictures for Compton scattering.}
\label{fig:compton}
\end{figure}
The idea of the experiment is to beam photons of known momentum $Q$ at
a target of stationary electrons, and measure the momenta $Q'$ of the
scattered photons as a function of scattering angle $\theta$.  We
therefore want to derive an expression for $Q'$ as a function of
$\theta$.

Before the collision the 4-momenta of the photon and electron are
\begin{equation}
\fv{p}_{\gamma} = (Q,Q,0,0)\nonumber\\
\end{equation}
\begin{equation}
\fv{p}_e = (m\,c,0,0,0)
\end{equation}
respectively, and after they are
\begin{equation}
\fv{q}_{\gamma} = (Q',Q'\,\cos\theta,Q'\,\sin\theta,0)
\end{equation}
\begin{equation}
\fv{q}_e = (\gamma\,m\,c,\gamma\,m\,v\,\cos\phi,-\gamma\,m\,v\,\sin\phi,0)
\end{equation}
respectively, where we have aligned coordinates so the initial
direction of the photon is the $x$-direction, and the scatter is in
the $x$-$y$ plane.  The conservation law is
\begin{equation}
\fv{p}_{\gamma}+\fv{p}_e=\fv{q}_{\gamma}+\fv{q}_e
\end{equation}
but there is a trick.  We can move both the photon 4-momenta to one
side and both the electron momenta to the other and square (where
$\fv{a}^2$ is just $\fv{a}\cdot\fv{a}$):
\begin{equation}
(\fv{p}_{\gamma}-\fv{q}_{\gamma})^2 = (\fv{q}_e-\fv{p}_e)^2
\end{equation}
\begin{equation}
\fv{p}_{\gamma}\cdot\fv{p}_{\gamma}
+ \fv{q}_{\gamma}\cdot\fv{q}_{\gamma}
- 2\,\fv{p}_{\gamma}\cdot\fv{q}_{\gamma}
= \fv{p}_e\cdot\fv{p}_e + \fv{q}_e\cdot\fv{q}_e
- 2\,\fv{p}_e\cdot\fv{q}_e
\end{equation}
For all photons $\fv{p}\cdot\fv{p}=0$ and for all electrons
$\fv{p}\cdot\fv{p}=m^2\,c^2$.  Also, in this case,
$\fv{p}_{\gamma}\cdot\fv{q}_{\gamma}=Q\,Q'-Q\,Q'\,\cos\theta$ and
$\fv{p}_e\cdot\fv{q}_e=\gamma\,m^2\,c^2$, so
\begin{equation}
- 2\,Q\,Q'\,(1-\cos\theta) = 2\,(1-\gamma)\,m^2\,c^2
\end{equation}
But by conservation of energy, $(\gamma-1)\,m\,c$ is just $Q-Q'$, and
$(a-b)/ab$ is just $1/b-1/a$, so we have what we are looking for:
\begin{equation}
\frac{1}{Q'}-\frac{1}{Q} = \frac{1}{m\,c}\,(1-\cos\theta)
\end{equation}

This prediction of special relativity was confirmed in a beautiful
experiment by Compton (1923) and has been reconfirmed many times
since by undergraduates in physics lab courses.  In addition to
providing quantitative confirmation of relativistic mechanics, this
experimental result is a beautiful demonstration of the fact that
photons, though massless, carry momentum and energy.

Quantum mechanics tells us that the energy $E$ of a photon is related
to its frequency $\nu$ by $E=h\,\nu$, and we know that for waves
travelling at speed $c$, the frequency $\nu$ and wavelength $\lambda$
are related by $\lambda=c/\nu$, so we can re-write the Compton
scattering equation in its traditional form:
\begin{equation}
\lambda'-\lambda = \frac{h}{m\,c}\,(1-\cos\theta)
\end{equation}

% Problem: what is the maximum neutrino mass given SN 1987a?


\section{Mass transport by photons}
\label{sec:masstransport}

Consider a box of length $L$ and mass $m$ at rest on a frictionless
table.  If a photon of energy $E\ll m\,c^2$ is emitted from one end of
the box (as shown in Figure~\ref{fig:photonmass}) and is absorbed by
the other, what is the reaction of the box?
\begin{figure}
\horfigure{masstransport.eps}
\caption[A demonstration that there is a mass associated with the
energy in a photon]{A thought experiment to demonstrate that there is
a mass $\mu=E/c^2$ associated with a photon of energy $E$.}
\label{fig:photonmass}
\end{figure}

We know the previous section that a photon of energy $E$ carries
momentum $E/c$, so to conserve momentum, the emission of the photon
must cause the box to slide backwards at a speed $v$ given by
$m\,v=-E/c$ (where it is okay to use the classical formula $m\,v$ for
momentum because we stipulated $E\ll m\,c^2$ so $\gamma\ll 1$). The
photon is absorbed a time $\Delta t$ later, and the box must stop
moving (again to conserve momentum).  In time $\Delta t$, the box
moves a distance
\begin{equation}
\Delta x_{\rm b} = v\,\Delta t = -\frac{E}{m\,c}\,\Delta t
\end{equation}
and then stops, while the photon moves a distance
\begin{equation}
\Delta x_{\rm p} = c\,\Delta t = L-\frac{E}{m\,c}\,\Delta t
\end{equation}
and then gets absorbed.  Because the forces associated with the
emission and absorption of the photon are totally internal to be box,
we do not expect them to be able to transport the center of mass of
the box (see, e.g., Frautschi et al., 1986, Chapter~11 for a
non-relativistic discussion of this---it is a consequence of
conservation of momentum).  But because the box moved, the center of
mass can only have remained at rest if the photon transported some
mass $\mu$ from one end of the box to the other.  To preserve the
center of mass, the ratio of masses, $\mu/m$ must be equal to the
ratio of their displacements $\Delta x_{\rm b}/\Delta x_{\rm p}$, so
\begin{equation}
\mu = m\,\frac{\Delta x_{\rm b}}{\Delta x_{\rm p}}
= \frac{E}{c^2}
\end{equation}
The transmission of the photon thus transports a mass $\mu=E/c^2$.

This does not mean that the photon is massive.  The {\em rest mass\/}
of a photon is zero.  It only shows that when a photon of energy $E$
is emitted, the emitter loses mass $\Delta m=E/c^2$ and when it is
absorbed the absorber gains mass $\Delta m=E/c^2$.

\begin{problem}
In Chapter~\ref{chap:causality} we learned that no signal can travel
through a solid body at a speed faster than that of light.  The part
of the box which absorbs the photon, therefore, won't know that a
photon has been emitted from the other end until the photon actually
arrives\footnote{I acknowledge French (1966) for pointing out this
problem with the above argument.}!  Re-cast this argument for mass
transport by photons into a form which does not rely on having a box
at all.
\end{problem}


\section{Particle production and decay}
\label{sec:decay}

\begin{problem}
A particle of mass $M$, at rest, decays into two smaller
particles of masses $m_1$ and $m_2$.  What are their energies and
momenta?
\end{problem}

Before decay, the 4-momentum is $(E/c,\tv{p})=(Mc,\tv{0})$.  After,
the two particles must have equal and opposite 3-momenta $\tv{p}_1$
and $\tv{p}_2$ in order to conserve 3-momentum.  Define $p\equiv
|\tv{p}_1|=|\tv{p}_2|$; in order to conserve energy
$E_1+E_2=E=M\,c^2$ or
\begin{equation}
\sqrt{p^2+m_1^2\,c^2}+\sqrt{p^2+m_2^2\,c^2}=M\,c
\end{equation}
This equation can be solved (perhaps numerically---it is a quartic)
for $p$ and then $E_1=\sqrt{m_1^2c^4+p^2c^2}$ and
$E_2=\sqrt{m_2^2c^4+p^2c^2}$.

\begin{problem}
Solve the above problem again for the case $m_2=0$.  Solve the
equations for $p$ and $E_1$ and then take the limit $m_1\rightarrow
0$.
\end{problem}

\begin{problem}
If a massive particle decays into photons, explain using
4-momenta why it cannot decay into a single photon, but must decay
into two or more.  Does your explanation still hold if the particle is
moving at high speed when it decays?
\end{problem}

\begin{problem}
A particle of rest mass $M$, travelling at speed $v$ in the
$x$-direction, decays into two photons, moving in the positive and
negative $x$-direction relative to the original particle.  What are
their energies?  What are the photon energies and directions if the
photons are emitted in the positive and negative $y$-direction
relative to the original particle (i.e., perpendicular to the
direction of motion, in the particle's rest frame).
\end{problem}


\section{Velocity addition (revisited) and the Doppler shift}
\label{sec:vadd2}

The fact that the 4-momentum transforms according to the Lorentz
transformation makes it very useful for deriving the velocity addition
law we found in Section~\ref{sec:vadd}.  In frame $\rfr{S}$, a
particle of mass $m$ moves in the $x$-direction at speed $v_1$, so its
4-momentum is
\begin{equation}
\fv{p}=(\gamma_1\,m\,c, \gamma_1\,m\,v_1, 0, 0)
\end{equation}
where $\gamma_1\equiv (1-v_1^2/c^2)^{-1/2}$.  Now switch to a new
frame $\rfr{S}'$ moving at speed $-v_2$ in the $x$-direction.  In this
frame the 4-momentum is
\begin{eqnarray}
\fv{p}' & = &
\left(\gamma_2\,\gamma_1\,m\,c +
\frac{v_2}{c}\,\gamma_2\,\gamma_1\,m\,v_1,\right.
\nonumber\\
& & \left.\gamma_2\,\gamma_1\,m\,v_1 + \frac{v_2}{c}\,\gamma_2\,\gamma_1\,m\,c,
0, 0\right)
\end{eqnarray}
The speed is just the ratio of $x$ and $t$-components, so
\begin{eqnarray}
\frac{v'}{c} & = & \frac{
\gamma_2\,\gamma_1\,m\,v_1 + \gamma_2\,\gamma_1\,m\,v_2}{
\gamma_2\,\gamma_1\,m\,c + \gamma_2\,\gamma_1\,m\,v_1\,v_2/c}
\nonumber \\
v' & = & \frac{v_1+v_2}{1+v_1\,v_2/c^2}
\end{eqnarray}
This is a much simpler derivation than that found in Section~\ref{sec:vadd}!

Consider now a photon in $\rfr{S}$ with 4-momentum $\fv{q}=(Q,Q,0,0)$.
In frame $\rfr{S}'$ the 4-momentum is
\begin{equation}
\fv{q}'=(\gamma_2\,Q+\frac{v_2}{c}\,\gamma_2\,Q,
\gamma_2\,Q+\frac{v_2}{c}\,\gamma_2\,Q,0,0)
\end{equation}
Clearly this is still travelling at the speed of light (as it must)
but now its new 3-momentum is
\begin{equation}
Q'=\gamma_2\,\left(1+\frac{v_2}{c}\right)\,Q=\sqrt{\frac{1+v_2}{1-v_2}}\,Q
\end{equation}
This change in momentum under a boost is the {\em Doppler shift,} and
is discussed in more detail in the next Chapter.


\section{4-force}
\label{sec:4-force}

We now have 4-velocity and 4-momentum, and we know how to use them.
If we want to construct a complete, invariant dynamics, analogous to
Newton's laws but valid in all reference frames, we are going to need
4-acceleration and 4-force.  Recall that we defined a 4-vector to be a
four component object that transforms according to the Lorentz
transformation.  For this reason, the 4-velocity $\fv{u}$ and
4-momentum $\fv{p}$ are defined in terms of derivatives with respect
to proper time $\tau$ rather than coordinate time $t$.  The
definitions are $\fv{u}\equiv d\fv{x}/d\tau$ and $\fv{p}\equiv
m\fv{u}$, where $\fv{x}$ is spacetime position and $m$ is rest mass.

For this same reason, if we want to define a 4-vector form of
acceleration, the 4-acceleration $\fv{a}$, or a 4-vector force,
4-force $\fv{K}$, we will need to use
\begin{eqnarray}
\fv{a} & \equiv & \frac{d\fv{u}}{d\tau} \\
\fv{K} & \equiv & \frac{d\fv{p}}{d\tau}
\end{eqnarray}
Because $\fv{p}=(E,\tv{p})$, we have
\begin{equation}
\fv{K} = \left(\frac{dE}{d\tau},\frac{d\tv{p}}{d\tau}\right) \; .
\end{equation}
Because $\Delta t=\gamma\,\Delta\tau$ (where, as usual, $\gamma\equiv
(1-v^2/c^2)^{-1/2}$), the spatial part of the 4-force is related to
Newton's force $\tv{F}$, defined as $\tv{F}\equiv d\tv{p}/dt$, by
\begin{equation}
\frac{d\tv{p}}{d\tau} = \gamma\,\tv{F}
\end{equation}
Also, if the rest mass $m$ of the object in question is a constant
(not true if the object in question is doing work, because then it
must be using up some of its rest energy!), we have that
\begin{eqnarray}
\fv{p}\cdot\fv{p} & = & m^2\,c^2\nonumber\\
\frac{d}{d\tau}(\fv{p}\cdot\fv{p}) & = & 0\nonumber\\
\frac{d\fv{p}}{d\tau}\cdot\fv{p} + \fv{p}\cdot\frac{d\fv{p}}{d\tau}
 & = & 0\nonumber\\
\fv{p}\cdot\fv{K} & = & 0
\end{eqnarray}
i.e., if the rest mass is not changing then $\fv{p}$ and $\fv{K}$ are
{\em orthogonal.\/}  In 3+1-dimensional spacetime, orthogonality is
something quite different from orthogonality in 3-space: it has
nothing to do with $90\degree$ angles.

The 4-force is only brought up here to whet the reader's appetite.  We
will actually have to make use of it in the (currently non-existent)
Chapter on electricity.

% Problem: uniformly accelerated astronaut.