\chapter{Generalization to general relativity}
\label{chap:general}

This chapter does not rely heavily on the material in the previous
three chapters (mechanics, apparent effects, and electromagnetism),
except in one or two places.

Recall Einstein's principle of relativity: the laws of physics (and in
particular the speed of light) should be the same for all observers.
In Section~\ref{sec:approximations} we restricted this principle to
only those observers in ``freely falling'' reference frames;
i.e. weightless observers or observers who experience no gravity.
This restriction gives special relativity the ``special.''  Einstein
sought to include all observers in a more general theory, and this
effort resulted in {\em general relativity,\/} one of the most
beautiful theories in modern physics.  Although general relativity
goes far beyond the scope of this tiny chapter, it is possible, by the
use of simple thought experiments, to derive some of the elementary
results.


\section{Einstein's principle of equivalence}

In constructing his general theory of relativity, Einstein sought to
make fundamental an apparent coincidence which appears in Newton's
theory of gravity.\footnote{The reasoning is set out quite simply in
Einstein (1911).}  In Newton's system, the acceleration $\tv{a}$ of a
body is proportional to the force $\tv{F}$ acting upon it, with the
mass $m$ the constant of proportionality, i.e., $\tv{F}=m\,\tv{a}$.
Technically, in this context, $m$ is the {\em inertial mass,\/} the
quantity responsible for ``resisting acceleration.''  In a
gravitational field $\tv{g}$, the gravitational force $\tv{F}_g$
acting on a body is $m\,\tv{g}$.  In this context, $m$ is the {\em
gravitational mass,\/} the quantity responsible for ``producing''
gravitational force.  The ``coincidence'' of the inertial mass
equalling the gravitational mass for all objects has the simple result
that every free body in the gravitational field undergoes precisely
the same acceleration.  Einstein was unsatisfied by the coincidence
explanation and sought to raise this observed symmetry up to a
fundamental postulate of physics.

Einstein suggested the {\em equivalence principle:\/} there is no
local experiment\ldots


\section{Gravitational deflection of light}

It is simple to show that an observer standing in a gravitational
field will see light travel not on straight paths (in the Euclidean
sense) but on curved ones, as if light feels gravitational attraction.


\section{Gravitational redshift}

Photons lose energy as they climb out of a gravitational potential
well.  That is, they become lower-frequency or longer-wavelength or
are shifted to the red.  There are two simple arguments to show this.

The first is very simple but relies on the equivalence of mass and
energy as described in Section~\ref{sec:masstransport}.\footnote{I
believe that this argument was first put forward by Einstein (1911).}
Imagine that we have two ideal machines.  One, the {\em photon
absorber,\/} absorbs photons and converts each photon's energy $E$
into mass, creating a particle of mass $m=E/c^2$.  The other, the {\em
photon emitter\/} absorbs massive particles and converts each
particle's total energy $E$ (rest mass energy plus kinetic energy)
into a photon of the same total energy, which is emitted.  Although no
such machines exist yet, it is at least physically possible for such
mechanical devices to have behaviours approaching those of the ideal
machines.

\begin{figure}
\vspace{2in}
\caption[An experiment to demonstrate gravitational redshift]{}
\label{fig:tower}
\end{figure}
Now consider a tall diving tower of height $h$ on the surface of the
Earth, where the gravitational field has strength $g$.  Place the
photon absorber at the top of the tower and the photon emitter at the
bottom, as shown in Figure~\ref{fig:tower}.  Begin by having the
photon emitter emit a photon of energy $E_0$, in the direction of the
absorber.  Imagine, contra reality, that the photon makes it up to the
absorber with no reduction in energy.  At the top the photon is
converted into a particle of mass $m=E_0/c^2$.  The particle is
dropped back down to the emitter, when it lands it has kinetic energy
$m\,g\,h$ because it has fallen through gravitational potential
difference $g\,h$.  On landing, the particle has total energy
\begin{equation}
E_1 = m\,c^2 + m\,g\,h = E_0\,\left(1+\frac{g\,h}{c^2}\right) \;.
\end{equation}
The photon emitter converts this particle into a photon of energy
$E_1$ which travels back up to the absorber.  The absorber drops a new
particle of mass $E_1/c^2$ which lands with energy
\begin{equation}
E_2 = E_1\,\left(1+\frac{g\,h}{c^2}\right)
= E_0\,\left(1+\frac{g\,h}{c^2}\right)^2 \;.
\end{equation}
Clearly, although there is no energy input to the machines, the
energies of the photons keep increasing.  This violates conservation
of energy.

Conservation of energy can only be respected if the photon emitted
with energy $E_0$ is absorbed at the top of the tower with energy
\begin{equation}
E_{\rm abs} = \frac{E_0}{1+gh/c^2} \;,
\end{equation}
that is photons must lose energy as they climb out of the
gravitational potential well.  In general, if the photon travels up
through a gravitational potential difference $\Delta\Phi$ from when it
is emitted with energy $E_{\rm em}$ to when it is absorbed, it will be
absorbed with energy
\begin{equation}
E_{\rm abs} = \frac{E_{\rm em}}{1+\Delta\Phi/c^2}
\end{equation}
In Section~\ref{sec:redshift}, we introduced redshift $z$ by an
equation of the form
\begin{equation}
E_{\rm abs} = \frac{E_{\rm em}}{1+z} \;,
\end{equation}
so the redshift produced in a photon travelling through gravitational
potential difference $\Delta\Phi$ is $z=\Delta\Phi/c^2$.  Just like
the kinematic redshift, this gravitational redshift affects frequency,
so $\nu_{\rm abs}=\nu_{\rm em}/(1+z)$, and wavelength, so
$\lambda_{\rm abs}=(1+z)\,\lambda_{\rm em}$.  An interesting
consequence is that two observers, at rest with respect to one another
but at different heights in a gravitational potential will observe
each others' clocks to run at a different speed than their own.  The
lower observer sees the upper's clock running fast, or blueshifted, and
the upper sees the lower's clock running slow, or redshifted.

The second argument for gravitational redshift is more difficult but
does not depend on the equivalence of mass and energy or the mass
transport of photons.  Recall that the laws of special relativity only
strictly hold for freely falling reference frames and consider two
such frames.  The first, $\rfr{S}$, is a freely falling frame which is
instantaneously at rest with an observer at the bottom of a diving
tower at the exact moment that he or she emits a photon.  The second,
$\rfr{S}'$, is a freely falling frame which is instantaneously at rest
with an observer at the top of the tower at the exact moment at which
he or she receives the photon.  The frames are both freely falling,
but they are at rest with respect to the two observers (and hence the
tower and the Earth) at different times.  $\rfr{S}$ is at rest with
respect to the Earth at the time of emission and $\rfr{S}'$ at
reception, a time $h/c$ later.  In that time, $\rfr{S}$ has acquired a
downward speed relative to the Earth of $g\,h/c$ and is therefore
moving relative to $\rfr{S}'$ at speed $g\,h/c$.  Special relativity
holds in these freely-falling frames, and special relativity tells us
that a signal beamed from one observer to another will be received
with a redshift or blueshift depending on their relative speeds
(Section~\ref{sec:redshift}), in the limit of small $v$, in this case
small $h$, the redshift is $z\approx v/c$ or in this case $z\approx
g\,h/c^2$.  When this problem is done correctly, including all the
time dilation and length contraction that is involved, the exact
answer comes out $z=\Delta\Phi/c^2$.

In 19??, ?? performed an incredibly sensitive experiment which
successfully measured the gravitational redshift\ldots

Gravitational redshift is also detected in the emission from near the
surface of compact, massive objects such as white dwarf stars\ldots


\section{Problems with Newton's gravity}

Potential $\Phi$ is $G\,M\,m/r$, but what distance $r$ do you use?
The length $r$ is not a scalar.  At what time do you evaluate $\Phi$,
which depends on the spatial distribution of matter do you use?  Do
you need a retarded potential?

Also, there is this puzzle about the inertial and gravitational mass.
Why are they so similar, when there is no reason within Newton's
theory of gravity itself to suggest this equivalence.


\section{Freely falling reference frames}

A reminder that SR only truly holds in freely falling reference
frames.  Throw a ball while standing on the Earth, throw one while
falling from a diving tower.  In the former case the ball travels on a
curved path, a parabola.  In the latter case the trajectory is a
straight line.


\section{The metric and curved spacetime}

For advanced or enthusiastic readers.

