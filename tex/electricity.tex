\chapter{Electricity and magnetism}
\label{chap:electricity}

Special relativity was initially conceived by Einstein as a way to
unify electricity and magnetism, so our treatment of the subject is
historically backwards; we have only just now reached the {\em
beginning\/} of the story.  Before Einstein's time, Maxwell had
written down his beautiful equations relating the electric and
magnetic fields $\tv{E}$ and $\tv{B}$ (reference ??).  Maxwell's
equations showed incredible symmetry between $\tv{E}$ and $\tv{B}$,
but the two fields remained physically distinct.  Hertz had shown that
accelerating charges radiate waves in these fields, waves which travel
at the speed of light (reference ??).  And Lorentz had noticed that
the electric field of a rapidly-moving charge was flattened along the
direction of motion relative to the spherical field produced by a
static charge (reference ??).  Einstein's great accomplishment was to
explain or at least unify electricity and magnetism, showing that the
electric field requires a magnetic field counterpart if the speed of
light in vacuum and all the laws of physics are the same in all
uniformly moving reference frames.

The goal of this chapter is to use special relativity and
electrostatic laws to ``predict'' or demonstrate the existence and
properties of the magnetic field.  Some knowledge of electrostatics is
assumed, although not much.  Purcell (1985), Chapter~1, is recommended
to any readers who feel a bit rusty or inexperienced with
electrostatics.


\section{Conservation and invariance of charge}

Charge is {\em conserved.\/} That is, in any reaction, interaction, or
collision, the sum of the charges of all the components {\em before\/}
equals the sum of all charges of all the components {\em after.\/}
Charge is also {\em invariant.\/} That is, every object has the same
charge in all frames or for all observers.  Note that the first law,
conservation, is a statement about physical processes whereas the
second, invariance, is about reference frames.

In support of charge conservation, the evidence comes in the form of the
millions of collisions and scattering events observed in particle
accelerators around the world; no violation of charge conservation has
ever been observed.\footnote{Although a violation is predicted
at??}

Charge invariance can be argued in a very simple way with the
principle of relativity.  The principle of relativity states that the
fundamental physical constants and laws are the same for all observers
in all inertial reference frames.  This must include the charge on the
electron (or proton).  In addition, we cannot change the fact of an
object's existence merely by changing frames.  Thus, neither the
charges on the fundamental particles, nor the numbers of those
particles can depend on reference frame.  The total charge on any
object must be independent of reference frame.

The above argument depends on the linearity of the equations of
electromagnetism, which I will merely assert (it is discussed in
detail in Purcell 1980), and it doesn't say anything about the
electric field.  Just because the mathematical quantity obtained by
multiplying the charge on an electron times the number of excess
electrons in an object is frame-independent, it does not follow that
the Gauss's law integral of electric flux through a closed surface
enclosing the object is also frame-independent.  One extremely
compelling reason to believe that it {\em is} frame-independent is the
electric neutrality of (i.e. lack of long-range electric fields from)
stationary atoms, in which the nucleus is at rest while the electrons
move at speeds on the order of $10^{-2}\,c$.  This neutrality is known
to hold to at least one part in ?? and it is independent of atomic
motion.  Neither of these facts would hold if the Gauss's law electric
flux integral was not proportional to charge, independent of reference
frame.


\section{The electric field}

In a sense you were cheated in the previous section, because I did not
tell you how charge is {\em defined.}  It is meaningless to say that
charge is invariant and conserved if we do not say what it is!  The
definition of charge is tied to the definition of the electric field,
and the definition of the electric field is tied to the concept of
charge.  Don't expect these notes to sort it all out.

The following definition of the electric field $\tv{E}$ will be
assumed in these notes.  If at some position a small test charge of
charge $q$ which is {\em at rest\/} feels an electrostatic 3-force
$\tv{F}$, the (3-vector) electric field $\tv{E}$ at that position is
defined to be $\tv{E}\equiv \tv{F}/q$.  The requirement that the
charge be at rest is important, because the magnetic field will cause
forces that are proportional to speed.

In the cgs system, the electric field of a point charge $Q$ is
\begin{equation}
\tv{E}=\frac{Q}{r^2}\,\hat{\tv{r}} \;\;,
\end{equation}
where $r$ is the distance from the charge, and $\hat{\tv{r}}$ is the
radial unit 3-vector.  For more general situations the electric field
can be computed using Gauss's law which relates the field $\tv{E}$ to
the charge density $\rho$ (charge per unit volume)
\begin{equation}
\int_{\cal S} \tv{E}\cdot d\tv{a} = 4\pi\,\int_{\cal V} \rho\,dV
= 4\pi\,Q_{\rm enclosed} \;\;,
\end{equation}
where the first integral is a surface integral of the electric flux
through a closed surface $\cal S$ and the second integral is a volume
integral which gives the total charge $Q_{\rm enclosed}$ inside the
volume $\cal V$.  The differential form of Gauss's law is
\begin{equation}
\tv{\nabla}\cdot\tv{E} = 4\pi\,\rho \;\; ,
\end{equation}
where $\tv{\nabla}\cdot$ is the 3-vector divergence operator.

Actually Gauss's law is the definition of charge.  That is, the charge
$Q_{\rm enclosed}$ inside volume $\cal V$ is {\em defined\/} to be
$1/(4\pi)$ times the electric flux through the bounding surface $\cal
S$.\footnote{We defined $\tv{E}$ with force on a test charge, and
charge with a surface integral of $\tv{E}$.  We are running in
circles, but this is nothing new for us physicists.  Recall, for an
example from elementary mechanics, Newton's second law
$\tv{F}=m\,\tv{a}$.  This equation simultaneously defines the
undefined quantities 3-force $\tv{F}$ and (inertial) mass $m$ in terms
of one another.  Much worse, the equation is only true in certain
reference frames, inertial ones, but an inertial frame is defined to
be a frame in which the equation holds!  The amazing thing is that
despite all this, the laws of physics do have predictive power.}  The
flux integral must be performed at a particular time, that is the
integral over the surface of $\tv{E}$ is the value of $\tv{E}$ at one
particular time $t$.  The inferred charge is the charge contained
inside the volume at that time $t$.


\section{Boost transforming a pure electric field.}

The simplest application of Gauss's law is the analysis of parallel
plate capacitors; this and what we know about length contraction will
teach us the first fact we need in constructing the magnetic field.
Consider the parallel plate capacitor shown in
Figure~\ref{fig:capacitor1}, at rest in the lab (unprimed) frame.  It
is composed of two large plates of area $A$, carrying uniformly
distributed charges $+Q$ and $-Q$.  The plates are separated by a
small distance $h$.  Align coordinates so the plates are parallel to
the $x$-$y$ plane.
\begin{figure}
\vspace{2in}
\caption[A parallel plate capacitor at rest]{A parallel plate
capacitor at rest in the lab frame, with the Gauss's law ``pillbox''
surface used to calculate the electric field shown.}
\label{fig:capacitor1}
\end{figure}
If the plates are much larger in extent than the distance between
them, edge effects can be ignored.  In this approximation, symmetry
requires that the electric field point in the $z$ direction, and that
its magnitude depend only on $z$.  If a ``pillbox'' Gauss's law
surface of cross-sectional area $a$ is placed so that it cuts through
one plate as shown in Figure~\ref{fig:capacitor1}, it will contain
total charge $q_{\rm enclosed}=a\,Q/A$, so a change in the magnitude
of $\tv{E}$ of $4\pi\,Q/A$ is expected in going from one side of the
plate to the other.  This argument does not depend on the height of
the pillbox, so the electric field must be constant everywhere except
for discontinuities of magnitude $4\pi\,Q/A$ at each plate.  The only
solution which preserves the symmetry has $\tv{E}=\tv{0}$ outside and
$\tv{E}=4\pi\,Q/A\,\hat{\tv{z}}$ inside, as shown in
Figure~\ref{fig:capacitor2}.
\begin{figure}
\vspace{2in}
\caption[The field of a parallel-plate capacitor]{The field of a
parallel plate capacitor, in the approximation that edge effects can
be ignored.}
\label{fig:capacitor2}
\end{figure}

Now consider that same capacitor, but in the rocket (primed) frame,
moving at speed $v$ in the $x$-direction.  By length contraction
(Chapter~\ref{chap:time}) its $x$-length is shorter by a
factor of $\gamma$ (recall that $\gamma\equiv (1-\beta^2)^{-1/2}$ and
$\beta\equiv v/c$), while its $y$-length and $z$-height are
unaffected.  So the area of the capacitor is $A'=A/\gamma$.  Gauss's
law defines charge and charge is invariant, so the same method for
calculating the electric field can be applied.  In the rocket frame,
the field inside the capacitor is $\tv{E}'=4\pi\,Q/A'\,\hat{\tv{z}}'$
which is a factor of $\gamma$ stronger than $\tv{E}$ in the lab frame.

Now we can turn the experiment through $90\degrees$ and look at a
parallel-plate capacitor at rest in the lab but with plates parallel
to the $y$-$z$ plane (Figure~\ref{fig:capacitor3}).  The field inside
is $\tv{E}=4\pi\,Q/A\,\hat{\tv{x}}$ in the lab frame.  In the rocket
frame the distance between the plates is reduced by a factor of
$\gamma$, but the area stays the same.  So $\tv{E}'=\tv{E}$; the
electric field is the same in both frames.
\begin{figure}
\vspace{2in}
\caption[A parallel plate capacitor at $90\degrees$]{A parallel plate
capacitor (a)~at rest in the lab frame and (b)~moving in the rocket
frame, but this time normal to the direction of motion.}
\label{fig:capacitor3}
\end{figure}

We have derived transformation laws for the field of a parallel plate
capacitor when that field is perpendicular to and parallel to the
direction of motion.  What happens when we have a general electric
field, not necessarily produced by parallel plates?  Because the
relationship between charge and the electric field, Gauss's law, has
the same form in all frames, any transformation of the electric field
produced by one charge distribution must be valid for all charge
distributions that could produce that electric field.  That is, the
transformation properties of the field are independent of the source
charge distribution.  This is required for the laws of
electromagnetism to have the same form in all frames.


\section{Electric field of a moving charge}

Given these transformation laws for electric field parallel to and
perpendicular to the direction of motion, it is possible to derive the
electric field of an isolated moving charge.

The electric field $\tv{E}$ of an electric point charge $Q$ at rest in
frame $\rfr{S}$ is given by
\begin{equation}
\tv{E}=\frac{Q}{r^2}\,\hat{\tv{r}} \; ,
\end{equation}
where $r$ is the distance of the charge from the position in question
and $\hat{\tv{r}}$ is the unit 3-vector which points radially away
from the center of the charge, which can also be taken to be the
center of the coordinate system.  At a point $(x,y,z)$ in cartesian
coordinates the field has the form
\begin{equation}
\tv{E}= \frac{Q}{r^2}\,\hat{\tv{r}}
      = \frac{Q\,x}{r^3}\,\hat{\tv{x}}
      + \frac{Q\,y}{r^3}\,\hat{\tv{y}}
      + \frac{Q\,z}{r^3}\,\hat{\tv{z}}
\end{equation}
Now switch to a frame $\rfr{S}'$ in which the charge is moving in the
$z$-direction at speed $v=\beta c$, and consider the point in
$\rfr{S}'$ corresponding to $(x,y,z)$ in $\rfr{S}$ at the moment
$t'=0$ at which the charge is also at the center of the $\rfr{S}'$
coordinate system.  The point corresponding to $(x,y,z)$ will be at
$(x',y',z')=(x,y,z/\gamma)$ by length contraction, where, as usual,
$\gamma\equiv (1-\beta^2)^{-1/2}$.  The magnitude of the
$z'$-component of $\tv{E}'$ will be unaffected, but the $x'$- and
$y'$-components will be increased by a factor of $\gamma$ so
\begin{eqnarray}
\tv{E}' & = & \frac{\gamma\,Q\,x}{r^3}\,\hat{\tv{x}}
            + \frac{\gamma\,Q\,y}{r^3}\,\hat{\tv{y}}
            + \frac{Q\,z}{r^3}\,\hat{\tv{z}} \nonumber\\
        & = & \frac{\gamma\,Q\,x'}{r^3}\,\hat{\tv{x}}'
            + \frac{\gamma\,Q\,y'}{r^3}\,\hat{\tv{y}}'
            + \frac{\gamma\,Q\,z'}{r^3}\,\hat{\tv{z}}'
\end{eqnarray}
where $\tv{E}'$ has been left in an ugly form because it still makes
use of $r$, the distance in frame $\rfr{S}$.
\begin{eqnarray}
r^2 &=& x^2+y^2+z^2 \nonumber\\
    &=& x'^2+y'^2+\gamma^2\,z'^2 \nonumber\\
    &=& r'^2\,\left[1+(\gamma^2-1)\,cos^2\theta'\right] \nonumber\\
    &=& r'^2\,\gamma^2\,\left[1-\beta^2\,sin^2\theta'\right]
\end{eqnarray}
where $\theta'$ is the angle from the $z'$ axis so
$z'/r'=\cos\theta'$.  Putting it together,
\begin{equation}
\tv{E}'= \frac{Q}{r'^2\,\gamma^2\,(1-\beta^2\,sin^2\theta')^{3/2}}\,\hat{\tv{r}}'
\end{equation}
There are several remarkable things about this expression.  The first
is that the field is radial everywhere, i.e., it points directly away
from the center of the charge.  Another is that the field is symmetric
front--back.  Neither of these properties are directly required by
either the laws of electromagnetism or those of special relativity.

The reader may note that this derivation was somewhat ugly, making use
of length contraction and a particular coordinate system.  My strong
intuition is that there is a much more elegant derivation which makes
use of the Maxwell tensor (introduced below), which, unlike the
electric field \tv{E}, is a frame-independent, geometric object.


\section{Force from a current-carrying wire}

Einstein's principle of special relativity says that the laws of
physics must be the same for all uniformly moving observers.  It turns
out (as will be seen below) that when the invariance of charge is
combined with our earlier special relativity results, we make
different predictions for the electric forces on moving charges in
different frames.  If relativity is to be saved, there needs to be a
velocity-dependent force which compensates for this; it will turn out
to be the {\em magnetic force,\/} and the truly invariant laws of
electric charges are the laws of {\em electromagnetism,\/} not simply
electricity.

Consider a long, thin wire, stationary and parallel to the $x$-axis in
frame $\cal{S}$, carrying current.  The wire contains a linear charge
density $\lambda_e$ (charge per unit length) in electrons which are
moving through the wire at mean speed $\beta_ec$, and a linear charge
density $\lambda_i$ in ions which are stationary.  The wire is
neutral, so the total linear charge density
$\lambda=\lambda_i+\lambda_e=0$, but it does carry current $I$ (charge
per unit time) equal to $\lambda_e\beta_ec$.  In this frame the charge
feels no force from the wire because the wire is neutral and (as we
will see) because the charge is stationary.

Now consider a frame $\rfr{S}'$ moving at speed $v=\beta c$ in the
$x$-direction relative to frame $\rfr{S}$.  In $\rfr{S}'$ the ions are
moving at speed $\beta_i'c=-\beta c$ and, because of length
contraction, have increased charge density
$\lambda_i'=\gamma\lambda_i$, where $\gamma\equiv (1-\beta^2)^{-1/2}$.
In $\rfr{S}'$ the electrons are moving at speed
$\beta_e'c=(\beta_e-\beta)c/(1+\beta_e\beta)$ and, again because of
length contraction, have linear charge density
$\lambda_e'=\lambda_e\,\gamma_e'/\gamma_e$, where $\gamma_e\equiv
(1-\beta_e^2)^{-1/2}$ and $\gamma_e'\equiv (1-\beta_e'^2)^{-1/2}$.  It
is left as an exercise for the reader to show that
$\gamma_e'=\gamma\gamma_e(1-\beta\beta_e)$.  The total charge density
on the wire, $\lambda_i+\lambda_e$ is therefore not zero in
$\rfr{S}'$, it is
\begin{eqnarray}
\lambda' & = & \lambda_i'+\lambda_e' \nonumber \\
 & = & \gamma\,(\lambda_i+\lambda_e)
       - \gamma\,\beta\,\beta_e\,\lambda_e \nonumber \\
 & = & - \beta\,\frac{\gamma\,I}{c}
\end{eqnarray}
where in the last line the neutrality of the wire in $\rfr{S}$ and
$I=\lambda_e\beta_ec$ have been used.  That is, a wire that is
uncharged but carrying current in one frame will be charged in
another.  This result does not stand in contradiction to the
invariance of charge, it follows from it: the formulas for the
$\lambda_i'$ and $\lambda_e'$ are derived by assuming the amount of
charge is constant as the wire is length-contracted.

Now consider the forces acting on a test charge $q$ which is at rest
in frame $\rfr{S}'$ a distance $r$ from the wire.  It sees an electric
field $E'=2\lambda'/r$ and feels a 3-force $F'=qE'$ in the radial
direction.  In frame $\rfr{S}$, the charge is moving at speed
$v=\beta\,c$ parallel to an uncharged, current-carrying wire.  If
Einstein's postulate of special relativity is correct, i.e., the laws
of physics are the same in all reference frames, then the charge must
feel a radial force in this frame as well; i.e., currents cause forces
on moving charges.  Putting it together, the force $F$ in on the
moving charge in frame $\rfr{S}$ is
\begin{eqnarray}
F & = & \frac{F'}{\gamma} \nonumber \\
  & = & - q\,\beta\,\frac{2\,I}{c\,r}
\end{eqnarray}
where we have used the relationship between force components
perpendicular to the direction of the boost in different frames
(Section~??).

As for the direction of the force, consider the effects of the boost
on the positive and negative linear charge densities in the wire.  For
definiteness, imagine that the test charge is positive and moving in
the same direction as the current in frame $\rfr{S}$, i.e., opposite
to the direction the (negative) electrons are moving.  Boosting into
frame $\rfr{S'}$, where the charge is at rest, we find that the
positive linear charge density has increased by a factor of $\gamma$
(by length contraction) while the negative linear charge density has
changed by a factor $\gamma_e'/\gamma$, which is
$\gamma(1-\beta\beta_e)$.  The negative linear charge density is
increased more in this case because $\beta$ and $\beta_e$ have
opposite signs.  So a positive charge is attracted to the wire when it
moves in the same direction as the current.  A negative charge
traveling in the same direction is repelled, as is a positive charge
travelling in the opposite direction to the current.

Before special relativity, these forces between currents and moving
charges, called {\em magnetic\/} forces, were not seen as being
directly related to electric forces.  Einstein's great achievement was
to ``explain'' magnetic forces as simply electric forces in another
frame of reference.

\begin{problem}
Show that $\gamma_e'=\gamma\gamma_e(1-\beta\beta_e)$ in the above
discussion of the current-carrying wire.
\end{problem}


\section{Force between current-carrying wires}

The first consequence of the forces from currents on moving charges is
that current-carrying wires will exert forces on one another.
Consider two parallel wires separated by a perpendicular distance $b$,
carrying currents $I_1$ and $I_2$ in the same direction.  The wires
are electrically neutral but contain charge densities $\lambda_1$ and
$\lambda_2$ in electrons, moving at speeds $\beta_1\,c$ and
$\beta_2\,c$.  The one wire will feel a force per unit length from the
other
\begin{eqnarray}
\frac{dF_1}{d\ell} & = & \lambda_1\,\beta_1\,\frac{2\,I_2}{c\,b} \nonumber \\
 & = & \frac{2\,I_1\,I_2}{c^2\,b}
\end{eqnarray}
and the force will be attractive.

This force between current-carrying wires is actually used as the
original definition of the SI unit of charge, the {\em Ampere.\/} A
current of 1~A is the current which, when carried in opposite
directions by two parallel wires of length and separation 1~m,
produces a force between the wires of 1~N.  In SI units, a charge of
1~Coulomb is defined to be an Ampere for a second or
1~A\,s.\footnote{The attentive reader may notice that earlier it was
written that charge was defined in terms of the electric field it
produces, and now that charge is defined in terms of forces between
current-carrying wires.  Is this inconsistent?  Not really, because we
have derived the forces between current-carrying wires directly from
special relativity and the electric fields of static charge
distributions.}

[Check if this is still the definition of an Ampere??]


\section{The magnetic field}

We have seen that the laws of electrostatics plus the laws of special
relativity require velocity-dependent, magnetic forces. In this
section, we derive more properties of magnetic forces, in terms of a
very useful vector field, the {\em magnetic field.\/} The electric
3-vector field $\tv{E}$ and magnetic 3-vector field $\tv{B}$ are
defined\footnote{Oh no!  Another definition problem.  We defined the
charge in terms of the fields and now the fields in terms of the force
on a charge.  Is this inconsistent?  Not up to an overall scaling
factor, which is set by the definition of current in terms of force.}
in terms of the 3-force $\tv{F}$ on a test charge $q$ moving at
3-velocity $\tv{v}$ by
\begin{equation}
\tv{F}=q\,\left[\tv{E}+\frac{\tv{v}\times\tv{B}}{c}\right]
\end{equation}
where ``$\times$'' denotes the cross or vector product.\footnote{For a
refresher on vector operators, see ??}

The definition of the magnetic field can be combined with the forces
on a moving test charge near a current-carrying wire to get the field
of a current-carrying wire.  A charge moving parallel to the current
but below the wire is pulled upwards; a charge moving the same way but
above is pulled down; on the left is pulled right and on the right is
pulled left.  Therefore the magnetic field $\tv{B}$ must describe
circles around the wire.  Given that the velocity and field vectors
are perpendicular, the magnetic field $B(r)$ as a function of
perpendicular distance $r$ from a long straight wire carrying current
$I$ is
\begin{equation}
B=\frac{2\,I}{c\,r}
\end{equation}
The direction of the field is given by a ``right-hand''
rule.\footnote{Those readers lacking right hands are advised to apply
the equivalent ``left-hand'' rule and then apply a negative sign to
their answers.}  If you curl the fingers of your right hand around the
wire such that they point in the direction of the magnetic field, your
thumb will point in the direction of the current.

The fact that the magnetic field obeys a right-hand rule does not
imply that the Universe has a handedness; it originates in the
definition of the cross product.  If the cross product had been
defined to use a left-hand rule, the magnetic field would also.

\begin{problem}
Use a symmetry argument to show that the magnetic field must describe
circles around the wire rather than helixes.  Helixes would also
satisfy the constraints from charges moving parallel to the current,
so you will have to consider charges moving in directions other than
those parallel to the wire.
\end{problem}


\section{Boost transforming electric and magnetic fields}

In a boost transformation, as we shall see, the electric and magnetic
fields ``mix'' in much the same way as time and space, or the
components of a 4-vector.  Neither the electric nor magnetic fields
are invariants, although there is a scalar combination of them which
is an invariant.  The simplest description of the fields is in terms
of a single second-rank tensor field, rather than two 3-vector fields,
but that description is postponed to near the end of this chapter
because it is somewhat difficult to construct initially.

[Get magnetic field given electric.  Get electric given magnetic.
Summarize and show that $E^2-B^2$ is invariant.]


\section{The charge--current-density 4-vector}

Charge is invariant and conserved.  Microscopically, charge
conservation is ensured by the equations of particle physics.
Macroscopically, charge conservation is ensured by a differential
equation acting on $\rho$, the charge density, which is thought of as
a field filling all of space.  The standard way to write the
conservation is
\begin{equation}
\frac{\partial\rho}{\partial t}+\tv{\nabla}\cdot(\rho\,\tv{v})=0
\end{equation}
where the first term is the local rate of change in the charge density
and the second term is the divergence of the flow of
charge.\footnote{Again, for a refresher on vector operators, see ??}

[Use the above to motivate the 4-d divergence and the
charge--current-density 4-vector.  Show that the 4-vector is indeed a
4-vector.]


\section{The invariance of the laws of electromagnetism}

Fiddle around to get the whole covariant picture.  Show that 4-force
is the same in all frames.


\section{The Maxwell tensor}

Do it all again with the Maxwell tensor for advanced readers.


\section{Field of an accelerated charge}

To conclude this chapter we turn to a subject which is quite
different, concerned with {\em changes\/} in the state of motion of a
charge.  Imagine that a charge is moving past an observer along a
straight path at constant speed.  The electric field measured by the
observer, as we have learned, is not spherically symmetric, but is in
fact compressed or contracted along the line of sight.  Now imagine
that when the charge is one meter away from the observer, it suddenly
comes to a halt for some reason (for instance imagine that it collides
with something).  The earliest the observer can possibly learn about
the halting is one meter of time {\em after\/} it actually comes to a
halt.  After the information about the field (which is also carried by
the field) gets to the observer, the observer will see the field of a
charge which is stationary at the point at which the charge stopped.

The above scenario goes as follows: At times $ct<0$, a charge $q$ is
traveling along the $x$-axis in the positive direction at speed $v$.
An observer sitting at rest at $x=z=0$ but $y=1$~m observes the field
to be radial and directed towards the charge's {\em current\/}
position, as was shown in Section~??.  At time $ct=0$ the charge
suddenly stops at the origin.  At times $0<ct<1$~m, the charge is in
fact at rest, but the observer doesn't know this; the observer
observes the field continue to point radially towards the point at
which the charge {\em would\/} be, had it not changed its state of
motion.  Only at times $ct>1$~m has the information in the electric
field about the stopping of the charge reached the observer; when
$ct>1$~m the observer observes the field to be the spherically
symmetric and pointing towards the origin.  At time $ct=1$~m, the
observer experiences a rapid jump in electric field strength and
direction (accompanied, as will be shown, by a spike or pulse in field
strength).  The scenario is illustrated in Figure~??.

[Insert figure here.]

Why does Figure~?? show the electric field lines connected inside the
thickness of the ``stopping horizon''?  Recall that the divergence of
the electric field is equal to the charge density, by Gauss's law.
This means that in any small volume inside the stopping horizon (or
anywhere else except where the charge is), the field lines ``going
in'' must also ``come out''.  The field lines only start and stop on
charges.  The consequence is that field lines get bunched up inside
the thickness of the stopping horizon.  Recall also that in the field
line picture, the strength of the field is proportional to the density
of the field lines.  The result is an electric field pulse inside the
stopping horizon.  The observer sees the field of a moving charge,
then a huge electric pulse and then the field of a stopped charge.

The ambitious reader will note (and subsequently derive quantitative
expressions for) several things about Figure~??.  One is that there is
a great deal of field strength in the pulse in the directions
perpendicular to the original direction of motion, and very little
along and opposite to the original direction of motion.  Another is
that although the field strength in the pulse decreases as the radius
of the horizon expands, it does not decrease as rapidly as $1/r^2$,
the radial dependence of the field from a uniformly moving charge.
