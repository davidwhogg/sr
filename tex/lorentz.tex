\chapter{The Lorentz transformation}
\label{chap:lorentz}

In this Chapter the invariant interval is introduced and the Lorentz
transformation is derived and discussed.  There is a lot of algebra
but it is straightforward and the results are simple.  The ``twin
paradox'' is explained in terms of geodesics.


\section{Proper time and the invariant interval}
\label{sec:interval}

In 3-dimensional space, two different observers can set up different
coordinate systems, so they will not in general assign the same
coordinates to a pair of points $P_1$ and $P_2$. However they will
agree on the {\em distance\/} between them.  If one observer measures
coordinate differences $\Delta x$, $\Delta y$ and $\Delta z$ between
points $A$ and $B$, and another, with a different coordinate system,
measures $\Delta x'$, $\Delta y'$ and $\Delta z'$, they will both
agree on the total distance $\Delta r$, defined by
\begin{eqnarray}
(\Delta r)^2 & \equiv & (\Delta x)^2 + (\Delta y)^2 + (\Delta z)^2 \nonumber\\
             & = & (\Delta x')^2 + (\Delta y')^2 + (\Delta z')^2 \; .
\end{eqnarray}
We would like to find a similar quantity for pairs of events: some
kind of `length' in 3+1-dimensional spacetime that is
frame-independent, or the same for all observers.  There is such a
quantity, and it is called the {\em invariant interval\/} or simply
{\em interval,\/} it is symbolized by $(\Delta s)^2$ and defined by
\begin{eqnarray}
(\Delta s)^2 & \equiv & (c\,\Delta t)^2 - (\Delta r)^2 \nonumber\\
& \equiv & (c\,\Delta t)^2 - (\Delta x)^2 - (\Delta y)^2 - (\Delta z)^2 \; ,
\end{eqnarray}
where $\Delta t$ is the difference in time between the events, and
$\Delta r$ is the difference in space or the distance between the
places of occurence of the events.

To demonstrate this, recall Section~\ref{sec:timedilation} in which we
considered the flashes of a lightclock carried by D.  In D's frame the
flashes are separated by time $c\Delta t=1~{\rm m}$ and distance
$\Delta x=0$.  The interval between flashes is therefore $(\Delta
s)^2=(c\Delta t)^2-(\Delta x)^2=1~{\rm m^2}$.  In E's frame $c\Delta
t'=\gamma\,(1~{\rm m})$ and $\Delta x=\gamma\,u\,(1~{\rm m})/c$, so
the interval is $(\Delta s')^2=\gamma^2\,(1-u^2/c^2)\,(1~{\rm m^2})$.
Since $\gamma\equiv (1-u^2)^{-1/2}$, $(\Delta s')^2=(\Delta
s)^2=1~{\rm m^2}$.  Any other observer moving at any other speed $w$
with respect to D will measure different time and space separations,
but a similar argument will show that the interval is still $1~{\rm
m^2}$.

The {\em proper time\/} $\Delta\tau$ between two events is the time
experienced by an observer in whose frame the events take place at the
same point if there is such a frame.  As the above example shows, the
square root of the invariant interval between the two events is $c$
times the proper time, or $c\,\Delta\tau=\sqrt{(\Delta s)^2}$. The
proper time is the length of time separating the events in D's frame,
a frame in which both events occur at the same place.  If the interval
is positive, there always is such a frame, because positive interval
means $|c\,\Delta t|>|\Delta\tv{r}|$ so a frame moving at vector
velocity $\tv{v}=(\Delta\tv{r})/(\Delta t)$, in which the events take
place at the same point, is moving at a speed less than that of light.

If the interval between two events is less than zero, i.e., $(\Delta
s)^2<0$, it is still invarant even though there is no frame in which
both events take place at the same point.  There is no such frame
because necessarily it would have to move faster than the speed of
light.  To demonstrate the invariance in this case, consider the
clock-synchronizing procedure described in
Section~\ref{sec:relofsimul}: two flashes are emitted together from a
point halfway between the clocks, separated by one meter.  The clocks
start when the flashes arrive, two events which are simultaneous in
their rest frame.  In the rest frame the two starting events are
separated by $c\,\Delta t=0$ and $\Delta x=1~{\rm m}$.  The interval is
$(\Delta s)^2=-1~{\rm m^2}$.  In the frame moving at speed $u$ with
respect to the rest frame, the clocks are separated by $(1~{\rm
m})/\gamma$ and they are moving so the light takes time $(0.5~{\rm
m})/[\gamma(c+u)]$ to get to one clock and $(0.5~{\rm m})/[\gamma(c-u)]$
to get to the other so $c\,\Delta t'$ is $c$ times the difference
between these, or $\gamma\,u\,(1~{\rm m})/c$.  Light travels at $c$ so
the displacement $\Delta x'$ is the $c$ times the sum, or
$\gamma\,(1~{\rm m})$.  The interval is $(\Delta s')^2=-1~{\rm m^2}$,
same as in the rest frame.  Since any other relative speed $w$ could
have been used, this shows that the interval is invariant even if it
is negative.

Sometimes the {\em proper distance\/} $\Delta\lambda$ is defined to be
the distance separating two events in the frame in which they occur at
the same time.  It only makes sense if the interval is negative, and
it is related to the interval by $\Delta\lambda=\sqrt{|(\Delta
s)^2|}$.

Of course the interval $(\Delta s)^2$ can also be exactly equal to
zero.  This is the case in which $(c\,\Delta t)^2=(\Delta r)^2$, or in
which the two events lie on the worldline of a photon.  Because the
speed of light is the same in all frames, a interval equal to zero in
one frame must equal zero in all frames.  Intervals with $(\Delta
s)^2=0$ are called ``lightlike'' or ``null'' while those with $(\Delta
s)^2>0$ are called ``timelike'' and $(\Delta s)^2<0$ are called
``spacelike''.  They have different causal properties, which will be
discussed in Chapter~\ref{chap:causality}.


\section{Derivation of the Lorentz transformation}
\label{sec:derive}

It would be nice to have algebraic formulae which allow us to compute
the coordinates $(ct',x',y',z')$ of an event in one frame given the
coordinates $(ct,x,y,z)$ of the event in some other frame.  In this
section we derive these formulae by assuming that the interval is
invariant and asking ``what kind of boost transformation will preserve
the interval?'', making one or two appeals to common sense on the way.

We want to find the linear\footnote{The reader may ask: why need the
transformation be linear?  It needs to be linear because straight
worldlines (i.e.\ constant-velocity worldlines) in one frame must
transform into straight worldlines in all other frames.}
transformation that takes the coordinates $(ct,x,y,z)$ of a
4-displacement in frame $\rfr{F}$ to the coordinates $(ct',x',y',z')$
it has in frame $\rfr{G}$ so that the interval is invariant and
$\rfr{G}$ is moving at speed $u=\beta\, c$ in the $x$-direction with
respect to $\rfr{F}$.

In Section~\ref{sec:lengthcontraction}, we argued that there are no
length distortions in the directions perpendicular to the direction of
motion.  This means that the $y$- and $z$-coordinates of an event in
$\rfr{F}$ must be the same as those in $\rfr{G}$;
\begin{eqnarray}
y' & = & y \nonumber\\
z' & = & z
\end{eqnarray}

Linearity requires that the $x'$ and $t'$ components must be given by
\begin{eqnarray}
c\,t' & = & L_{t't}\, c\,t + L_{t'x}\, x \nonumber\\
x' & = & L_{x't}\, c\,t + L_{x'x}\, x
\end{eqnarray}
where the $L_{i'j}$ are constants; or, in matrix\footnote{For a review
of matrix algebra, see the excellent textbook by Strang (1976).  In
short, a column vector multiplied by a matrix makes another column
vector according to the rule
$$
\left(\begin{array}{c} y_1\\ y_2\\ y_3\\ y_4 \end{array}\right) =
\left(\begin{array}{cccc}
a_{11}&a_{12}&a_{13}&a_{14}\\
a_{21}&a_{22}&a_{23}&a_{24}\\
a_{31}&a_{32}&a_{33}&a_{34}\\
a_{41}&a_{42}&a_{43}&a_{44}
\end{array}\right)\,
\left(\begin{array}{c} x_1\\ x_2\\ x_3\\ x_4 \end{array}\right)
$$
$$
\;\;= \left(\begin{array}{c}
a{11}x_1+a_{12}x_2+a_{13}x_3+a_{14}x_4 \\
a{21}x_1+a_{22}x_2+a_{23}x_3+a_{24}x_4 \\
a{31}x_1+a_{32}x_2+a_{33}x_3+a_{34}x_4 \\
a{41}x_1+a_{42}x_2+a_{43}x_3+a_{44}x_4
\end{array}\right)
$$
This is easily generalized to larger or smaller dimensions.} form,
\begin{equation}
\left(\begin{array}{c} c\,t' \\ x' \end{array}\right) =
\left(\begin{array}{cc} L_{t't} & L_{t'x} \\
L_{x't} & L_{x'x}\end{array}\right)
\,
\left(\begin{array}{c} c\,t \\ x \end{array}\right)
\end{equation}

From the previous chapter, we know that two events that occur in
$\rfr{F}$ at the same place (so $\Delta x=0$) but separated by time
$c\,\Delta t$ occur in $\rfr{G}$ separated by time $c\,\Delta
t'=\gamma\, c\,\Delta t$ and therefore separated in space by $\Delta
x'=-\beta\,c\,\Delta t'=-\beta\,\gamma\, c\,\Delta t$, where, as usual
$\gamma\equiv(1-\beta^2)^{-1/2}$.  This implies
\begin{eqnarray}
L_{t't} & = & \gamma \nonumber\\
L_{x't} & = & \gamma\,\beta
\end{eqnarray}

We also know that between any two events, the interval $\Delta s^2$ is
the same in all frames.  When $\Delta y=\Delta z=0$, $(\Delta
s)^2=(c\,\Delta t)^2-(\Delta x)^2$.  Combined with the above two
matrix elements, the requirement that $(\Delta s)^2=(\Delta s)'^2$
implies
\begin{eqnarray}
L_{t'x} = - \gamma\,\beta \nonumber\\
L_{x'x} = \gamma
\end{eqnarray}

So we find that the transformation of the coordinates from one frame
$\rfr{F}$ to another $\rfr{G}$ that is moving in the $x$-direction at
relative speed $+u=\beta\,c$ is given by
\begin{equation}
\left(\begin{array}{c} c\,t' \\ x' \\ y' \\ z' \end{array}\right) =
\left(\begin{array}{cccc}
\gamma        & - \gamma\,\beta & 0 & 0 \\
- \gamma\,\beta & \gamma        & 0 & 0 \\
0             & 0             & 1 & 0 \\
0             & 0             & 0 & 1
\end{array}\right)\,
\left(\begin{array}{c} c\,t \\ x \\ y \\ z \end{array}\right)
\end{equation}


\section{The Lorentz transformation}
\label{sec:lt}

The Lorentz transformation (hereafter LT) is very important and
deserves some discussion.  The LT really transforms differences
$(c\,\Delta t,\Delta x,\Delta y,\Delta z)$ between the coordinates of
two events in one frame to differences $(c\,\Delta t',\Delta x',\Delta
y',\Delta z')$ in another frame.  This means that if one is going to
apply the LT directly to event coordinates, one must be very careful
that a single event is at the origin $(0,0,0,0)$ of both
frames.\footnote{There is a more general class of transformations,
{\em Poincar\'e tranformations,\/} which allow translations of the
coordinate origin as well LTs (which include boosts and, as we will
see, rotations).}  In the previous section, we placed event $P$ at the
origin of both frames.

A simple consistency check we could apply to the LT is the following:
If we boost to a frame moving at $u$, and then boost back by a speed
$-u$, we should get what we started with.  In other words, LTs with
equal and opposite speeds should be the inverses of one another.  If
we change $u\rightarrow -u$, we have $\beta\rightarrow -\beta$ and
$\gamma\rightarrow\gamma$, so boosting the coordinates $(ct',x')$ in
frame $\rfr{K}$ back to $\rfr{H}$ and giving the new coordinates
double-primes, we have
\begin{eqnarray}
ct'' & = & \gamma\, ct' + \beta\,\gamma\, x' \nonumber \\
     & = & \gamma\, (\gamma\, ct - \beta\,\gamma\, x) 
           + \beta\,\gamma\, (- \beta\,\gamma\, ct + \gamma\, x) \nonumber \\
     & = & \gamma^2\, (ct - \beta\, x - \beta^2\, ct + \beta\, x) \nonumber \\
     & = & \gamma^2\, (1 - \beta^2)\, ct \nonumber \\
     & = & ct \\
 x'' & = & \beta\,\gamma\, ct' + \gamma\, x' \nonumber \\
     & = & \beta\,\gamma\, (\gamma\, ct - \beta\,\gamma\, x)
           + \gamma\, (- \beta\,\gamma\, ct + \gamma\, x) \nonumber \\
     & = & \gamma^2\, (\beta\, ct - \beta^2\, x - \beta\, ct + x) \nonumber \\
     & = & \gamma^2\, (1 - \beta^2)\, x \nonumber \\
     & = & x \; ,
\end{eqnarray}
so indeed, the boost of $-u$ is the inverse of the boost of $u$.

The LT as defined above has the primed frame ($\rfr{K}$) moving at
speed $+u$ with respect to the unprimed frame ($\rfr{H}$).  This is
not a universal convention, but I will try to stick to it.

The group of all LTs includes all linear transformations that preserve
the interval\footnote{In fact, the astute reader will notice that
there are linear transformations which preserve the interval but
involve reversing the direction of time or reflecting space through a
plane.  These do indeed satisfy the criteria to be LTs but they are
known as ``improper'' LTs because they do not correspond to physically
realizable boosts.  On the other hand, they do have some theoretical
meaning in relativistic quantum mechanics, apparently.}.  This means
that LTs include space rotations with no boost, for example
\begin{equation}
\left(\begin{array}{cccc}
1 & 0 & 0 & 0 \\
0 & \cos\theta & -\sin\theta & 0 \\
0 & \sin\theta & \cos\theta & 0 \\
0 & 0 & 0 & 1
\end{array}\right)
\end{equation}
LTs also include boosts in arbitrary directions, not just the
$x$-direction.  For an arbitrary relative velocity
$\tv{u}=(u_x,u_y,y_z)$ of frame $\rfr{S}'$ with respect to $\rfr{S}$,
the corresponding LT is
\begin{equation}
\left(\begin{array}{cccc}
\gamma           & -\gamma\,\beta_x & -\gamma\,\beta_y & -\gamma\,\beta_z \\
-\gamma\,\beta_x
  & 1+\frac{(\gamma-1)\,\beta_x^2}{\beta^2}
  & \frac{(\gamma-1)\,\beta_x\,\beta_y}{\beta^2}
  & \frac{(\gamma-1)\,\beta_x\,\beta_z}{\beta^2} \\
-\gamma\,\beta_y
  & \frac{(\gamma-1)\,\beta_x\,\beta_y}{\beta^2}
  & 1+\frac{(\gamma-1)\,\beta_y^2}{\beta^2}
  & \frac{(\gamma-1)\,\beta_y\,\beta_z}{\beta^2} \\
-\gamma\,\beta_z
  & \frac{(\gamma-1)\,\beta_x\,\beta_z}{\beta^2}
  & \frac{(\gamma-1)\,\beta_y\,\beta_z}{\beta^2}
  & 1+\frac{(\gamma-1)\,\beta_z^2}{\beta^2} \\
\end{array}\right)
\end{equation}
where we define
\begin{eqnarray}
& \beta_x \equiv u_x/c \nonumber\\
& \beta_y \equiv u_y/c \nonumber\\
& \beta_z \equiv u_z/c \nonumber\\
& \beta^2 \equiv \beta_x^2 + \beta_y^2 + \beta_z^2 \nonumber\\
& \gamma \equiv (1-\beta_x^2-\beta_y^2-\beta_z^2)^{-1/2}
\end{eqnarray}
(see, e.g., Jackson, 1975, Chapter~11).  And, of course, any
composition of arbitrary LTs is also an LT.

\begin{problem}
Transform the events $A$ $(ct,x)=(0,0)$, $B$ $(0,1\,{\rm m})$, $C$
$(1/2\,{\rm m},1/2\,{\rm m})$, $D$ $(1\,{\rm m},0)$, and $E$ $(1\,{\rm
m},1\,{\rm m})$ into a frame $\rfr{S}'$ moving at speed $+0.6c$ in the
$x$-direction with respect to the unprimed frame $\rfr{S}$.  Draw
spacetime diagrams of both frames showing the five events.
\end{problem}

To check your answer: notice that $A$, $C$, and $E$ all lie on a
$45\degree$ worldline, as do $B$, $C$, and $D$.  The LT must transform
$45\degree$ worldlines to $45\degree$ worldlines because the speed of
light is $c$ in all frames.

\begin{problem}
Write down the transformation from a frame $\cal S$ to a frame $\cal
S'$ moving at $+0.5c$ in the $x$-direction and then to another frame
$\cal S''$ moving at $+0.5c$ in the $x$-direction relative to $\cal
S'$.  What is the complete transformation from $\cal S$ to $\cal S''$?
What relative speed between frames $\cal S$ and $\cal S''$ does your
answer imply?
\end{problem}

\begin{problem}
Show that the transformations given for a coordinate rotation
and for a boost in an arbitrary direction preserve the interval.
\end{problem}

\begin{problem}
Do space reflections and time-reversals preserve the interval?
\end{problem}

\begin{problem}
Denote by $E$ the event on the $ct$-axis of a spacetime diagram
that is a proper time $c\tau$ from the origin.  What is the locus of
all events on the spacetime diagram that are separated from the origin
by the same proper time?
\end{problem}

The answer should be a hyperbola that asymptotes to the line $ct=x$
but which is horizontal on the spacetime diagram right at $E$.

\begin{problem}
Denote by $F$ the event on the $x$-axis of a spacetime diagram that is
a distance $\ell$ from the origin.  What is the locus of all events
which are separated from the origin by the same interval as $F$?
\end{problem}


\section{Velocity addition}
\label{sec:vadd}

We are now in a position to derive the correct velocity addition law
that replaces the simple but incorrect one suggested in
Section~\ref{sec:einstein}: If A moves at speed $+u$ in the
$x$-direction with respect to B, and A throws a cantaloupe at speed
$+v$ in the $x$-direction relative to himself, at what speed $w$ does
B observe the cantaloupe to travel?  The simple but incorrect answer
is $w=u+v$.  The correct answer can be quickly calculated with a
Lorentz transformation.  Call the throwing event $T$ and put it at the
origin of both frames, so $(ct_T,x_T)=(ct_T',x_T')=(0,0)$, where A's
frame gets the primes.  Now imagine that at some time $t'$ later in
A's frame, the cantaloupe explodes, this explosion event $E$ must
occur at coordinates $(ct',vt')$ in A's frame.  In B's frame, by
definition, $T$ occurs at the origin, but by applying the LT with
speed $-u$ (defining $\beta\equiv u/c$ and $\gamma$ accordingly) $E$
now occurs at
\begin{eqnarray}
ct & = & \gamma\, ct' + \beta\,\gamma\, vt' \nonumber\\
x & = & \beta\,\gamma\, ct' + \gamma\, vt'
\end{eqnarray}
The speed $w$ measured by B is simply $x/t$ or
\begin{eqnarray}
w & = & c\,\frac{\beta\,\gamma\, ct' + \gamma\, vt'}
                {\gamma\, ct' + \beta\,\gamma\, vt'} \nonumber\\
  & = & \frac{u + v}{1 + uv/c^2}
\end{eqnarray}
which is less than $u+v$.  Spacetime diagrams for this calculation are
shown in Figure~\ref{fig:vadd}.
\begin{figure}
\horfigure{vadd.eps}
\caption[Velocity addition illustrated.]{Spacetime diagrams of the
throw $T$ and explosion $E$ of C by A, as observed by (a)~A and (b)~B
for the purposes of computing the velocity addition law.}
\label{fig:vadd}
\end{figure}

\begin{problem}
In an interplanetary race, slow team X is travelling in their old
rocket at speed $0.9c$ relative to the finish line.  They are passed
by faster team Y, observing Y to pass X at $0.9c$.  But team Y
observes fastest team Z to pass Y's own rocket at $0.9c$.  What are
the speeds of teams X, Y and Z relative to the finish line?
\end{problem}

The answer is not $0.9c$, $1.8c$, and $2.7c$!

\begin{problem}
An unstable particle at rest in the lab frame splits into two
identical pieces, which fly apart in opposite directions at Lorentz
factor $\gamma=100$ relative to the lab frame.  What is one particle's
Lorentz factor relative to the other?  What is its speed relative to
the other, expressed in the form $\beta\equiv 1-\epsilon$?
\end{problem}

\begin{problem}
Determine the transformation law for an arbitrary 3-vector velocity
$\tv{v}=(v_x,v_y,v_z)$.
\end{problem}


\section{The twin paradox}
\label{sec:twins}

Lin (L) and Ming (M) are twins, born at the same time, but with very
different genes: L is an astronaut who likes to explore outer space,
and M is a homebody who likes to stay at home on Earth and read
novels.\footnote{Do not regard this statement as a position on the
nature/nurture debate.}  When both L and M turn 20, L leaves on a
journey to a nearby star.  The star is $\ell=30$~light years away and
L chooses to travel out at speed $u=0.99 c$ and then immediately turn
around and come back.  From M's point of view, the journey will take
time $T=2\ell /u\approx 60~{\rm yr}$, so L will return when M is 80.
How much will L have aged over the same period?

In Section~\ref{sec:timedilation} we learned that moving clocks go
slow, so L will have aged by $T'=T/\gamma$, where $\gamma\equiv
(1-\beta^2)^{-1/2}$ and $\beta\equiv u/c$.  For $u=0.99 c$,
$\gamma=7$, so L will have aged less than $9~{\rm yr}$.  That is, on
L's arrival home, M will be 80, but L will only be 28!  Strange, but
in this special relativistic world, we are learning to live with
strangeness.

During his journey, Ming starts to get confused about this argument.
After all, there is no preferred reference frame.  If one looks at the
Earth from the point of view of Ming's rocket, one sees the Earth
travel out at speed $u$ and come back.  So isn't L's clock the one
that runs slow, and won't L the one who will be younger upon return?
How can this be resolved?

In Figure~\ref{fig:twins}, the worldlines of L and M are plotted in
the rest frame of the Earth (frame $\rfr{S}$), with L's departure
marked as event $D$, L's turnaround at the distant star as $T$ and her
return home as $R$.
\begin{figure}
\horfigure{twins.eps}
\caption[Worldlines of the twins L and M.]{Worldlines of the twins L
and M in frame $\rfr{S}$, with L's departure marked as $D$, turnaround
as $T$ and return home as $R$.}
\label{fig:twins}
\end{figure}
You will recall that in Section~\ref{sec:interval} we saw that along a
worldline, the proper time, or time elapsed for an observer
travelling along the worldline, is the square root of the interval
$(\Delta s)^2=(c\,\Delta t)^2-(\Delta x)^2$.  M does not move, so
$\Delta x=0$ and the proper time for him is just $\Delta t_{DR}$.  L
moves very quickly, so $(\Delta x)$ is not zero, so her proper time
out to event $T$ and back again will be much smaller than simply
$\Delta t_{DR}$.  Smaller, of course, by a factor $1/\gamma$.

Let's draw this now in L's frame.  But we have a problem: just what
frame do we choose?  Do we choose the frame $\rfr{S'}$ that is L's
rest frame on her way out to the star, or the frame $\rfr{S''}$ that
is L's rest frame on the way back?  We cannot choose both because they
are different frames: L {\em changes frames\/} at event $T$.  This
breaks the symmetry and resolves the paradox: M travels from event $D$
to event $R$ in a single frame with no changes, while L changes
frames.  L's worldline is crooked while M's is
straight\footnote{Another, fundamentally incorrect, but nonetheless
useful, way to distinguish the twins is to imagine that despite their
genetic differences, they are both avid coffee drinkers.  If they each
spend the entire time between events $D$ and $R$ drinking coffee, L
experiences no trouble at all, but M finds that he spills his coffee
all over himself at event T.  After all, his spaceship suffers a huge
acceleration at that time.  L experiences no such trauma.  This
explanation is fundamentally flawed because if we allow for
gravitational forces, there are many ways to construct twin paradoxes
which do not involve this asymmetry.}.

It is easy to show that given any two events and a set of worldlines
that join them, the worldline corresponding to the path of longest
proper time is the straight line.  Just as in Euclidean space the
straight line can be defined as the shortest path between two points,
in spacetime the straight worldline can be defined as the path of
{\em longest proper time.\/}  This is in fact the definition, and
straight worldlines are called {\em geodesics.\/}

\begin{problem}
Prove that the straight worldline joining any two events $E$ and $F$
is the line of maximum proper time.  Hint: begin by transforming into
the frame in which $E$ and $F$ occur at the same place.
\end{problem}

\begin{problem}
Imagine that every year, on their respective birthdays, each twin
sends the other a radio message (at the speed of light).  Re-draw
Figure~\ref{fig:twins} on graph paper and draw, as accurately as
possible, L's birthday messages in red and M's birthday messages in
blue.  How many messages does each twin receive?  At what ages to they
receive them?
\end{problem}

\begin{problem}
Imagine that rather than taking one long trip out and back, Ming in
fact takes five shorter trips out and back, but all at the same speed
$\beta$, and elapsing the same total time (on Lin's clock) for all the
trips, as in the single-trip case.  What effect does this have on
Ming's aging relative to Lin's, as compared with the single-trip
case?  Estimate how much less a commercial airline pilot ages relative
to her or his spouse over her or his lifetime.
\end{problem}
