\chapter{Causality and the interval}
\label{chap:causality}

The sign of the interval $(\Delta s)^2$ (i.e., whether it is positive
or negative) is discussed in terms of causality in this Chapter.  If
one event can affect another causally, the interval between them must
be positive.  By preserving the interval, therefore, the Lorentz
transformation preserves also the causal structure of the Universe,
provided that nothing travels faster than light.  This is the reason
for that universal speed limit.


\section{The ladder and barn revisited}
\label{sec:ladder2}

Recall the ``ladder and barn'' paradox discussed in
Section~\ref{sec:ladder1}, in which N is at rest with respect to a
barn, and P is carrying a long ladder but running so that it will be
length contracted and therefore fit.

Confused by the discussion of relativity of simultaneity in
Chapter~\ref{chap:geometry}, N decides to {\em prove\/} that ladder
does indeed fit into the barn by replacing the back door with an
incredibly strong, rigid, and heavy back wall that does not open.  Now
when P enters the barn, he cannot leave, and the question is: does the
front door ever close at all?  If it closes, the ladder {\em must\/}
be really inside the barn in {\em all\/} frames because there is no
back door through which it can be exiting.  Thus instead of asking
whether event $C$ happens before or after $D$, a frame-dependent
question, we are asking whether $C$ happens at all.  This is a
frame-independent question.\footnote{Events are frame-independent
entities in the sense that if an event occurs in one frame, it must
occur in all.  One cannot ``undo'' the fact that one sneezed by
changing frames!  On the other hand, relationships between events such
as simultaneity are frame-dependent or relative.}

In N's frame, event $C$, the closing of the front door, {\em must\/}
happen because the front of the ladder does not hit the back wall
until event $C$ has occurred.  That is, the ladder does not even
``know'' that the back door has been replaced by a brick wall until
event $C$ has occurred, so if event $C$, the closing of the front
door, happened when the back door was open, it must still happen now
that the back door is no longer there.

In P's frame the front of the ladder hits the back of the barn before
the back of the ladder enters, as we saw in Section~\ref{sec:ladder1}.
But does this mean that the ladder will stop and event $C$ will no
longer happen?  To answer this question, we will have to actually do
some {\em Physics\/} for the first time in these notes.

If I am standing at one end of a long table of length $\ell$ and I
push on the table to move it, how quickly can someone standing at the
other end feel the table move?  My pushing on the table sets up a
compression wave that travels at the speed of sound $c_s$ in the
table.  The person at the other end feels the push when the wave gets
there, at a time $\ell /c_s$ after I push.  In everyday experience,
this time is fairly short, so we are not aware of the time delay
between the push at one end and the feeling at the other.  But if we
stand at opposite ends of a very long, stretched slinky, this time
delay is easily observable.

Because, as we will see, no object or piece of matter can ever travel
faster than the speed of light and because all information is
transferred via either matter or light itself, no information or
signal or, in particular, compression wave, can ever travel faster
than the speed of light.  This means that no matter how rigid and
strong I build my table, the earliest possible time that the person at
the other end can feel my push is at a time $\ell /c$ after I push,
where $c$ is now the speed of light.

Why this digression?  Because it applies to the problem at hand.
Sure, in P's frame, the front of the ladder hits the back of the barn
before the back of the ladder enters, but this information cannot
reach the back of the ladder until some finite time after the
collision.  So the back of the ladder doesn't know that anything has
gone awry at the front and it continues to move.  When does the back
of the ladder learn of the front's collision?  To answer this we need
to draw spacetime diagrams.  Figure~\ref{fig:ladder2} shows the
spacetime diagrams in the two frames.
\begin{figure}
\horfigure{ladder2.eps}
\caption[The ladder and barn revisited]{Same as
Figure~\ref{fig:ladder1} but now event $D$ is a collision rather than
an exit.  The news of the collision cannot travel faster than the
speed of light so it cannot reach the back of the ladder before event
$E$.}
\label{fig:ladder2}
\end{figure}
Event $D$ is the collision of the ladder with the back wall, and we
have added event $E$, the earliest possible moment at which the back
of the ladder can learn of the collision at the front.  This event is
separated from the collision event by a photon trajectory, because the
maximum speed at which the information can travel is the speed of
light.  In both frames we see that the back of the ladder enters the
barn and event $C$ occurs before the back of the ladder learns about
the collision.  In other words, the back of the ladder makes it into
the barn and the door closes behind it.  What does this imply?  It
implies that the ladder must be compressible or fragile.  The fact
that the speed of sound in the ladder cannot exceed the speed of light
ensures that all materials are compressible.  Loosely speaking, this
is because a totally incompressible substance has an infinite sound
speed, and that is not allowed.  There are many fun problems in
relativity based on this type of argument, discussion of which is
prevented by lack of space.  One important application is a proof that
dark (i.e., not burning nuclear fuel), compact objects more massive
than about three times the mass of the Sun must be black holes: any
other material, even a crystal composed of pure neutrons, can only
hold itself up under that kind of pressure if it is so rigid that the
speed of sound in the material would necessarily exceed the speed of
light!

\begin{problem}
Imagine a plank of length $\ell$ supported at both ends by sawhorses
in a gravitational field of acceleration $g$.  One support is kicked
out.  What is the minimum time the other end of the plank could
``know'' that the one end has lost its support?  Roughly speaking,
what distance $\Delta y$ will the one end fall before the other can
know?  How much does the board bend, and, to order of magnitude, what
does this tell you about, say, the Young's modulus of the board?
\end{problem}

\begin{problem}
Imagine a wheel of radius $R$ consisting of an outer rim of length
$2\pi\,R$ and a set of spokes of length $R$ connected to a central
hub.  If the wheel spins so fast that its rim is travelling at a
significant fraction of $c$, the rim ought to contract to less than
$2\pi\,R$ in length by length contraction, but the spokes ought not
change their lengths at all (since they move perpendicular to their
lengths).  How do you think this problem is resolved given the
discussion in this Section?  If you find a solution to this problem
which does not make use of the concepts introduced in this Section,
come see me right away!
\end{problem}


\section{Causality}
\label{sec:causality}

Event order is relative, but it is subject to certain constraints.  By
changing frames in the ladder-and-barn paradox, we can make event $D$
precede, be simultaneous with, or follow event $E$.  But we cannot
make {\em any\/} pair of events change their order simply by changing
frames.  For instance, if Quentin (Q) throws a ball to Rajesh (R), the
event of the throw $A$ must precede the event of the catch $B$ in all
frames.  After all, it is impossible for R to catch the ball before Q
throws it!

If indeed events $A$ and $B$ are the throwing and catching of a ball,
we can say something about their $x$ and $t$ coordinates.  The spatial
separation $\Delta x$ between the events must be less than the time
(in dimensions of distance) $c\,\Delta t$ between the events because
the ball cannot travel faster than the speed of light.  For such a
pair of events the interval
\begin{equation}
(\Delta s)^2 = (c\,\Delta t)^2 - (\Delta x)^2
\end{equation}
must be positive.  Events with positive interval must occur in the
same order in all frames because activity at the earlier event can
affect activity at the later event.  Such a pair of events has a {\em
timelike\/} spacetime separation, and it is sometimes said that $A$ is
in the {\em causal history\/} of $B$, or $B$ is in the {\em causal
future\/} of $A$.

In the case of events $C$ and $D$ in the ladder-and-barn paradox, the
interval between the events is negative, and any signal or information
or matter traveling between the events would have to travel faster
than the speed of light.  Thus activity at each of these events is
prevented from affecting activity at the other, so there is no logical
or physical inconsistency in having a boost transformation change
their order of occurence.  Such a pair of events has a {\em
spacelike\/} spacetime separation.  They are {\em causally
disconnected.\/}

For completeness we should consider events $D$ and $E$ in the
ladder-and-barn paradox.  These events are separated by a photon world
line or $(c\,\Delta t)^2=(\Delta x)^2$, so the interval is zero
between these events.  Such a pair of events is said to have a {\em
lightlike\/} or {\em null\/} spacetime separation.  Two events with a
null separation in one frame must be have a null separation in all
frames because the speed of light is the same in all frames.


\section{Nothing can travel faster than the speed of light}
\label{sec:speedlimit}
\markright{\thesection.~~NOTHING FASTER THAN LIGHT}

The well-known speed limit---nothing can travel faster than the speed
of light---follows from the invariant causal structure of the
Universe.  If one event is in the causal history of another in one
frame, it must be in that causal history in all frames, otherwise we
have to contend with some pretty wacky physics.\footnote{The reader
who objects that special relativity is already fairly wacky will be
ignored.}  For instance, reconsider the above example of Q and R
playing catch.  Imagine that Q and R are separated by $\ell$ in their
rest frame $\rfr{S}$, and Q throws the ball to R at twice the speed of
light.  The spatial separation between events $A$ and $B$ is $\Delta
x=\ell$ and the time separation is $c\Delta t=\ell/2$.  Now switch to
a frame $\rfr{S}'$ moving at speed $v$ in the direction pointing from
Q to R.  Applying the Lorentz transformation, in this frame
\begin{equation}
\Delta t'=\gamma\,\ell\,\left(\frac{1}{2}-v\right)
\end{equation}
which is less than zero if $v>1/2$.  In other words, in frames
$\rfr{S}'$ with $v>1/2$, event $B$ precedes event $A$.  I.e., in
$\rfr{S}'$ R must catch the ball before Q throws it.  A little thought
will show this to be absurd; we are protected from absurdity by the
law that nothing, no object, signal or other information, can travel
faster than the speed of light.  Thus we have been justified, earlier
in this chapter, and elsewhere in these notes, in assuming that
nothing can travel faster than the speed of light.
