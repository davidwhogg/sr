\chapter{The geometry of spacetime}
\label{chap:geometry}

Observers in different frames of reference, even if they are observing
identical events, may observe very different relationships {\em
between\/} those events.  For example, two events which are
simultaneous for one observer will not, in general, be simultaneous
for another observer.  However, the principle of relativity must hold,
i.e., both observers must agree on all laws of physics and in
particular on the speed of light.  This principle allows detailed
construction of the differences between two observers' measurements as
a function of their relative velocity.  In this chapter we derive some
of these relationships using a very useful tool: the spacetime
diagram.  With spacetime diagrams most special relativity problems are
reduced to simple geometry problems.  The geometric approach is the
most elegant method of solving special relativity problems and it is
also the most robust because it requires the problem-solver to
visualize the relationships between events and worldlines.


\section{Spacetime diagrams}
\label{sec:sd}

Frances (F) and Gregory (G) live on planets A and B, respectively,
separated in space by $\ell=6\times 10^{11}~{\rm m}$ (600 million km).
Exactly halfway between their home planets, on the line joining them,
is an interplanetary caf\'e (C), at which they decide to meet at noon.
F has a standard-model spaceship which travels at speed $c/5$ (which
corresponds to $\beta=1/5$), while G's sporty model travels at $c/3$
($\beta=1/3$).  If we choose a coordinate system with the $x$-axis
pointing along the direction from A to B, we can plot the
trajectories, or {\em worldlines,\/} of F and G on a diagram with
distance $x$ on the abscissa and time $t$ on the ordinate.  Actually,
to emphasize the geometry of special relativity, we will use not $t$
to mark time, but $ct$, which has dimensions of
distance.\footnote{Recall the idea, from
Section~\ref{sec:timedilation}, that $c$ is merely a conversion factor
between time and distance.}
\begin{figure}
\horfigure{cafe.eps}
\caption[Worldlines of F and G meeting at the caf\'e]{Worldlines of F
and G meeting at the caf\'e, and worldlines of their home planets A
and B, and the caf\'e itself, C.  The event of F's departure is $K$,
of G's is $L$, and of their meeting is $M$.  This diagram is in the
rest frame of A, B, and C because these objects have vertical
worldlines.  Note that the time (vertical) axis is marked in units of
distance $ct$.}
\label{fig:cafe}
\end{figure}
Such a plot, as in Figure~\ref{fig:cafe}, is a {\em spacetime
diagram.\/} Figure~\ref{fig:cafe} is clearly drawn in the rest frame
of planets A and B: the planet worldlines are vertical; the planets do
not change position with time.

They meet at noon at the caf\'e.  Their meeting is an {\em event:\/}
it takes place in a certain location, at a certain time.  Anything
that has both a position and a time is an event.  For example, the
signing of the United States' Declaration of Independence was an
event: it took place on 4 July, 1776, and it took place in
Philadelphia, Pennsylvania.  Each tick of a clock is an event: it
happens at a given time at the location of the clock.  Events are
3+1-dimensional\footnote{One could say ``4-dimensional,'' but it is
customary among relativists to separate the numbers of space and time
dimensions by a plus sign.  The reason for this will be touched upon
later.}  points---they have three spatial coordinates and one time
coordinate.  In the case of the meeting $M$ at the caf\'e of F and G,
we needed only 1+1 dimensions to specify it because we began by
restricting all activity to the $x$-axis, but in general 3+1
dimensions are needed.  On Figure~\ref{fig:cafe}, event $M$ is marked,
along with two other events $K$ and $L$, the departures of F and G.

Because we are marking time in dimensions of distance $ct$, the
inverse slope $\Delta x/(c\Delta t)$ of a worldline at some time $ct$
is the speed of the corresponding object in units of $c$, or in other
words, $\beta$.  As we will see below, nothing can travel faster than
the speed of light.  So, all worldlines must be steeper than
$45\degrees$ on the spacetime diagram, except, of course, for the
worldlines of flashes of light or photons, which have exactly
$45\degrees$ worldlines.  Radio, infrared, optical, ultraviolet, x-ray
and gamma-ray signals all travel on $45\degree$ worldlines maybe
neutrinos do too\footnote{As we will see in
Chapter~\ref{chap:mechanics}, neutrinos travel at the speed of light
only if they are massless; this is currently a subject of debate.}.

\begin{problem}
The next day F decides to meet G at the caf\'e again, but realizes
that she did not arrange this with G in advance.  She decides to send
a radio message that will get to G at exactly the time he should
depart.  When should F send this message?
\end{problem}

We can answer this problem trivially by looking at the spacetime
diagram.  If we drop a $45\degree$ line from event $L$, G's departure,
going back in time towards planet A, we can find the event at which it
intersects F's worldline.  This is done in
Figure~\ref{fig:whentosend}; we see that it intersects F's worldline
exactly at event $K$, the time of her departure.  This means that F
should send the radio message at exactly the time she departs for the
caf\'e.
\begin{figure}
\horfigure{cafe2.eps}
\caption[When should F send the radio message?]{When should F send the
radio message to G?  By dropping a $45\degree$ line (dotted) from
event $L$ to F's worldline, we find that she should send it right when
she departs; at event $K$.}
\label{fig:whentosend}
\end{figure}


\section{Boosting: changing reference frames}
\label{sec:boosting}

Heather (H) and Juan (J) are two more residents of planets A and B
respectively.  (A and B are separated by $\ell=6\times 10^{11}~{\rm
m}$ in the $x$-direction.)  Early in the morning (at event $P$) H
sends J a radio message.  At event $Q$, J receives the message.  A
time $\tau$ later in the day, H sends J another message at event $R$,
and J receives it at event $S$.  The spacetime diagram with these
events and the worldlines of H, J and the messages is shown in
Figure~\ref{fig:hjrest}.  The diagram is drawn in what we will call
``H's frame'' or ``H's rest frame,'' because it is a reference frame
in which H is at rest.
\begin{figure}
\horfigure{hjrest.eps}
\caption[Spacetime diagram for H, J, and the messages in H's rest
frame]{Spacetime diagram with worldlines of H, J, and the radio
messages (dotted), along with the sending and receiving events.  This
diagram is drawn in H's rest frame; her worldline is vertical.}
\label{fig:hjrest}
\end{figure}

While this is all going on, Keiko (K) is travelling at speed $u$
between planets A and B.  How do we re-draw the spacetime diagram in
K's frame, a reference frame in which K is at rest?  First of all, K
is moving at speed $u$ relative to H and J, so in K's frame H and J
will be moving at speed $-u$.  Thus, H's and J's worldlines in K's
frame will have equal but opposite slope to that of K's worldline in
H's frame.  Time dilation (Section~\ref{sec:timedilation}) says that
moving clocks go slow, so in K's frame, events $P$ and $R$ will be
separated in time not by $\tau$ but by $\Delta t'=\gamma\,\tau$.  Same
for $Q$ and $S$.  (All quantities in K's frame will be primed.)
Length contraction (Section~\ref{sec:lengthcontraction}) says that
moving ruler sticks are shortened.  This means that the distance
separating the parallel worldlines of two objects moving at the same
speed (the ``ends of the ruler stick'') is shorter by a factor
$1/\gamma$ in a frame moving at speed $u$ than it is in the frame at
which the two objects are at rest.  H and J, therefore, are separated
by not $\ell$ but $\Delta x'=\ell /\gamma$ in the horizontal
direction. Einstein's principle of relativity says that the speed of
light is the same in both frames, so {\em the radio signals will still
have $45\degree$ worldlines.\/} Thus, the spacetime diagram in K's
frame must be that pictured in Figure~\ref{fig:hjk}.
\begin{figure}
\horfigure{hjk.eps}
\caption[Spacetime diagram for H, J, and the messages in K's rest
frame]{Spacetime diagram with worldlines of H, J, and the radio
messages along with the sending and receiving events, now drawn in K's
rest frame.  Note the time dilation and length contraction.}
\label{fig:hjk}
\end{figure}

The transformation from H's frame to K's is a {\em boost
transformation\/} because it involves changing velocity.  The boost
transformation is central to special relativity; it is the subject of
this and the next chapter.

% Possible example problem: derive the full LT right here!

\begin{problem}
Re-draw the events and worldlines of Figures~\ref{fig:hjrest} and
\ref{fig:hjk} from the point of view of an observer moving at the same
speed as K relative to H and J but in the opposite direction.
\end{problem}

\begin{problem}
A rocket ship of proper length $\ell_0$ travels at constant speed $v$
in the $x$-direction relative to a frame $\cal S$.  The nose of the
ship passes the point $x=0$ (in $\cal S$) at time $t=0$, and at this
event a light signal is sent from the nose of the ship to the rear.
(a)~Draw a spacetime diagram showing the worldlines of the nose and
rear of the ship and the photon in $\cal S$.  (b)~When does the signal
get to the rear of the ship in $\cal S$?  (c)~When does the rear of
the ship pass $x=0$ in $\cal S$?  (After French 1966.)
\end{problem}

\begin{problem}
At noon a rocket ship passes the Earth at speed $\beta=0.8$.
Observers on the ship and on Earth agree that it is noon.  Answer the
following questions, and draw complete spacetime diagrams in both the
Earth and rocket ship frames, showing all events and worldlines:
(a)~At 12:30 p.m., as read by a rocket ship clock, the ship passes an
interplanetary navigational station that is fixed relative to the
Earth and whose clocks read Earth time.  What time is it at the
station?  (b)~How far from Earth, in Earth coordinates, is the
station?  (c)~At 12:30 p.m. rocket time, the ship reports by radio
back to Earth.  When does Earth receive this signal (in Earth time)?
(d)~The station replies immediately.  When does the rocket receive the
response (in rocket time)?  (After French 1966.)
\end{problem}


\section{The ``ladder and barn'' paradox}
\label{sec:ladder1}

Farmers Nettie (N) and Peter (P) own a barn of length $\ell$ and a
ladder of length $2\ell$.  They want to put the ladder into the barn,
but of course it is too long.  N suggests that P run with the ladder
at speed $u =0.866c$.  At this speed $\gamma =2$, so the ladder will
be shortened by enough to fit into the barn.  P objects.  P argues
that if he is running with the ladder, in {\em his\/} frame the ladder
will still have length $2\ell$ while the {\em barn\/} will be
shortened to length $\ell/2$.  The running plan will only make the
problem worse!

They cannot both be right.  Imagine P running with the ladder through
the front door of the barn and out the back door, and imagine that the
barn is specially equipped with a front door that closes immediately
when the back of the ladder enters the barn (event $C$), and a back
door that opens immediately when the front of the ladder reaches it
(event $D$).  Either there is a time when both doors are closed and
the ladder is enclosed by the barn, or there is not.  If there is such
a time, we will say that the ladder fits, and if there is not, we will
say that it does not fit.  Who is right?  Is N right that the ladder
is shorter and it will fit in the barn, or is P right that it isn't
and won't?

If we draw spacetime diagrams of the ladder and barn in both frames we
get Figure~\ref{fig:ladder1}, where the front and back of
the barn are labeled G and H respectively and the front and back of the
ladder are J and K respectively.
\begin{figure}
\horfigure{ladder1.eps}
\caption[The ladder and barn in both frames]{Worldlines of the front
and back of the barn (G and H) and the front and back of the ladder (J
and K) and events $C$ and $D$ in the rest frames of (a) N and (b) P.
While events $C$ and $D$ are simultaneous in N's frame, they are not
in P's.}
\label{fig:ladder1}
\end{figure}
In N's frame, indeed, events $C$ and $D$ are simultaneous, so there is
a brief time at which the ladder fits inside the barn.  In P's frame,
strangely enough, the events are no longer simultaneous!  Event $D$
happens long before event $C$, so there is no time at which the ladder
is entirely inside the barn.  So indeed both N and P are correct:
whether or not the ladder fits inside the barn is a frame-dependent
question; it depends on whether or not two events are simultaneous,
and {\em simultaneity is relative.\/}


\section{Relativity of simultaneity}
\label{sec:relofsimul}

How can we synchronize two clocks that are at rest with respect to one
another but separated by a distance $\ell$?  The simplest thing to do
is to put a lightbulb halfway between the two clocks, flash it, and
have each clock start ticking when it detects the flash.  The
spacetime diagram in the rest frame $\rfr{S}$ for this synchronizing
procedure is shown in Figure~\ref{fig:synch}, with the light bulb at
the origin and the two clocks at $x=\pm\ell/2$.  The flash is marked
as event $F$ and the detections of the flash as events $G$ and $H$.
Thereafter, the clock ticks are shown as marks on the clock worldlines.
\begin{figure}
\horfigure{synch.eps}
\caption[Synchronizing clocks in frame $\rfr{S}$.]{Synchronizing
clocks at rest in frame $\rfr{S}$ by flashing a lightbulb halfway
between them at event $F$ and having each clock start when it detects
the flash (events $G$ and $H$).  After the two clocks receive the
flashes, they tick as shown.  Lines of simultaneity connect
corresponding ticks and are horizontal.}
\label{fig:synch}
\end{figure}
Simultaneous ticks lie on horizontal lines on the spacetime diagram,
because they occur at the same value of the time coordinate.  In fact,
the horizontal lines can be drawn in; they are {\em lines of
simultaneity.\/}

Now consider a new frame $\rfr{S'}$ which is moving at speed
$+u=\beta\, c$ in the $x$-direction with respect to $\rfr{S}$.  In
this new frame, the worldlines of the clocks are no longer vertical
because they are moving at speed $-u$, but by Einstein's principle of
relativity the flashes of light must still travel on $45\degree$
worldlines.  So the spacetime diagram in $\rfr{S'}$ looks like
Figure~\ref{fig:synchprime}.
\begin{figure}
\horfigure{synchprime.eps}
\caption[Clocks in frame $\rfr{S'}$ that are synchronized in
$\rfr{S}$.]{The clocks as observed in frame $\rfr{S'}$ along with
events $F$, $G$, $H$, and the subsequent ticks.  Although the clocks
are synchronized in $\rfr{S}$ they are not in $\rfr{S'}$.  Note that
the lines of simultaneity (horizontal in $\rfr{S}$) are slanted in
$\rfr{S'}$.}
\label{fig:synchprime}
\end{figure}
Note that in $\rfr{S'}$ the lines of simultaneity joining the
corresponding ticks of the two clocks are no longer horizontal.  What
does this mean?  It means that two events which are simultaneous in
$\rfr{S}$ will not in general be simultaneous in $\rfr{S'}$.


\section{The boost transformation}
\label{sec:boosttx}

We have seen in the previous section that ``horizontal'' lines of
simultaneity in one frame become ``tilted'' in another frame moving
with respect to the first, but can we quantify this?  We can, and it
turns out that the lines of simultaneity in frame $\rfr{S}$ acquire
slope $-\beta$ in frame $\rfr{S'}$ (which moves at speed $+\beta\, c$
with respect to $\rfr{S}$) just as the lines of constant position in
$\rfr{S}$ acquire slope $-1/\beta$ in $\rfr{S'}$.  A simple thought
experiment to demonstrate this consists of two clocks, synchronized
and at rest in $\rfr{S}$, exchanging photons simultaneously in
$\rfr{S}$, as shown in Figure~\ref{fig:slope}.
\begin{figure}
\horfigure{slope.eps}
\caption[Clocks exchanging photons in their rest frame.]{Clocks at
rest and synchronized in frame $\rfr{S}$ exchanging photons.  They
emit photons simultaneously at events $A$ and $B$}
\label{fig:slope}
\end{figure}
In $\rfr{S}$ they emit photons simultaneously at events $A$ and $B$;
the photons cross paths at event $C$; and then are received
simultaneously at events $D$ and $E$.  In $\rfr{S'}$ events $A$ and
$B$ are no longer simultaneous, nor are events $D$ and $E$.  However,
light must still travel on $45\degree$ worldlines and the photons must
still cross at an event $C$ halfway between the clocks.  So the
spacetime diagram in $\rfr{S'}$ must look like
Figure~\ref{fig:slope2},
\begin{figure}
\horfigure{slope2.eps}
\caption[Clocks exchanging photons in another frame.]{Same as
Figure~\ref{fig:slope} but in frame $\rfr{S'}$.}
\label{fig:slope2}
\end{figure}
with the square $ABED$ in $\rfr{S}$ sheared into a parallelogram,
preserving the diagonals as $45\degree$ lines.  We know that the slope
of the lines of constant position transform to lines of slope
$-1/\beta$; in order to have the diagonals be $45\degree$ lines, we
need the lines of simultaneity to transform to lines of slope
$-\beta$.

This is really the essence of the boost transformation, the
transformation from one frame to another moving with respect to it:
the transformation is a shear or ``crunch'' along $45\degree$ lines.
A shear is a linear transformation that does not involve rotation, but
``squashes'' coordinates along one direction, allowing them to expand
along the perpendicular direction.  In this case, these directions are
photon trajectories or $45\degree$ worldlines.\footnote{The directions
along which the squash and expansion take place are the {\em
eigenvectors\/} of the transformation.  The ambitious reader is
invited to calculate the two corresponding eigenvalues.}  We will
derive the symbolic form of the boost transformation in
Chapter~\ref{chap:lorentz}, but for now these geometrical facts are
all we need.

\begin{problem}
Prove, using whatever you need (including possibly
Figures~\ref{fig:slope} and \ref{fig:slope2}), that if the clock world
lines have slope $1/\beta$ in some frame, the lines of simultaneity
will have slope $\beta$.  The shorter the proof, the better.
\end{problem}


\section{Transforming space and time axes}
\label{sec:txaxes}

One extremely useful way of representing the boost transformation
between two frames on spacetime diagrams is to plot the space and time
axes of both frames on both diagrams.  This requires us to utilize two
trivial facts: (a)~the spatial axis of a frame is just the line of
simultaneity of that frame which passes through the origin event
$(x,ct)=(0,0)$ and (b)~the time axis is just the line of constant
position which passes through $(0,0)$.  So if we (arbitrarily)
identify origin events in the two frames,\footnote{The zero of time
and space are arbitrary, so, with no loss of generality, we can assign
these values so that the origin events coincide.} we can plot, in
frame $\rfr{S'}$, in addition to the $x'$ and $ct'$ axes, the
locations of the $x$ and $ct$ axes of frame $\rfr{S}$
(Figure~\ref{fig:bothaxes}(a)).  We can also plot both sets of axes in
frame $\rfr{S}$.  This requires boosting not by speed $+\beta c$ but
rather by $-\beta c$ and, as you have undoubtedly figured out, this
slopes the lines in the opposite way, and we get
Figure~\ref{fig:bothaxes}(b).
\begin{figure}
\horfigure{bothaxes.eps}
\caption{Spacetime diagrams in frames (a)~$\rfr{S'}$ and
(b)~$\rfr{S}$, each showing the time and space axes of both frames.}
\label{fig:bothaxes}
\end{figure}
Again we see that the transformation is a shear.  Note that the boost
transformation is not a rotation, at least not in the traditional
sense of the word!

We are now in a position to answer the question posed at the end of
Section~\ref{sec:timedilation}: How can it be that two observers,
moving relative to one another, can {\em both} observe the other's
clock to tick more slowly than their own?  Imagine that observers at
rest in $\rfr{S}$ and $\rfr{S'}$ both draw lines of constant position
separated by 1~m of distance and lines of simultaneity separated by
1~m of time (3.3~ns) through the spacetime maps of their frames.  In
$\rfr{S}$, the $\rfr{S}$-observer's lines of constant position are
vertical, and lines of simultaneity are horizontal.  The
$\rfr{S'}$-observer's lines of constant position have slope $1/\beta$
and lines of simultaneity have slope $-\beta$, as seen in
Figure~\ref{fig:bothgrids}.
simultaneity.
\begin{figure}
\horfigure{bothgrids.eps}
\caption{Spacetime diagram in frame~$\rfr{S}$, showing the spacetime
grids drawn by the observer at rest in $\rfr{S}$ (solid) and the
observer at rest in $\rfr{S'}$ (dotted).  The $\rfr{S}$-observer's
clock ticks with solid dots and the $\rfr{S'}$-observer's with open
dots.  Note that when travelling along a dotted line of constant
position, clock ticks are encountered less frequently than solid lines
of simultaneity and when travelling along a solid line of constant
position, clock ticks are encountered less frequently than dotted
lines of simultaneity.  This explains how both observers can observe
the other's clock to run slow.}
\label{fig:bothgrids}
\end{figure}
In $\rfr{S}$, the horiztonal distance between the
$\rfr{S'}$-observer's lines of constant position is $(1~{\rm
m})/\gamma$.  Look carefully at Figure~\ref{fig:bothgrids}, which
shows the ticks of each observer's clock along a line of constant
position.  If we travel along the $\rfr{S'}$-observer's line of
constant position, we find that we encounter ticks of the $\rfr{S'}$
clock less frequently than lines of simultaneity in $\rfr{S}$.  On the
other hand, if we travel along the $\rfr{S}$-observer's line of
constant position, we find that we also encounter ticks of the
$\rfr{S}$ clock less freqently than lines of simultaneity in
$\rfr{S'}$.  That is, {\em both\/} observers find that the other's
clock is going slow.  There is no contradiction.

This point is subtle enough and important enough that the reader is
advised to stare at Figure~\ref{fig:bothgrids} until it is understood.

% Possible advanced section on how this IS a rotation but with cosh
% and sinh.  Also, it is possible to do the Lorentz boost with the
% ``generators'' of the Lorentz group.  See Leibovich for details.
