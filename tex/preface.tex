\section*{Preface}

For me, the wonder of special relativity lies in its successful
prediction of interesting and very nonintuitive phenomena from simple
arguments with simple premises.

These notes have three (perhaps ambitious) aims: (a)~to introduce
undergraduates to special relativity from its founding principle to
its varied consequences, (b)~to serve as a reference for those of us
who need to use special relativity regularly but have no long-term
memory, and (c)~to provide an illustration of the methods of
theoretical physics for which the elegance and simplicity of special
relativity are ideally suited.  History is a part of all science---I
will mention some of the relevant events in the development of special
relativity---but there is no attempt to present the material in a
historical way.

A common confusion for students of special relativity is between that
which is real and that which is apparent.  For instance, length
contraction is often mistakenly thought to be some optical illusion.
But moving things do not ``appear'' shortened, they actually {\em
are\/} shortened.  How they appear depends on the particulars of the
observation, including distance to the observer, viewing angles,
times, etc.  The observer finds that they are shortened only after
correcting for these non-fundamental details of the observational
procedure.  I attempt to emphasize this distinction: All apparent
effects, including the Doppler Shift, stellar aberration, and
superluminal motion, are relegated to Chapter~\ref{chap:astronomy}.  I
think these are very important aspects of special relativity, but from
a pedagogical standpoint it is preferable to separate them from the
basics, which are not dependent on the properties of the observer.

I love the description of special relativity in terms of
frame-independent, geometric objects, such as scalars and 4-vectors.
These are introduced in Chapter~\ref{chap:mechanics} and used
thereafter.  But even before this, the geometric properties of
spacetime are emphasized.  Most problems can be solved with a minimum
of algebra; this is one of the many beautiful aspects of the subject.

These notes, first written while teaching sections of first-year
physics at Caltech, truly represent a work in progress.  I strongly
encourage all readers to give me comments on any aspect of the
text\footnote{email: {\tt hogg@ias.edu}}; all input is greatly
appreciated.  Thank you very much.


\newpage
\section*{Acknowledgments}

Along with Caltech teaching assistantships, several NSF and NASA
grants provided financial support during the time over which this was
written.  I thank the enlightened members of our society who see fit
to support scientific research and I encourage them to continue.

My thanks go to the Caltech undergraduates to whom I have taught this
material; they shaped and criticized the content of these notes
directly and indirectly from beginning to end.  I also thank the
members of Caltech's astronomy and physics departments, faculty, staff
and my fellow students, from whom I have learned much of this
material, and Caltech for providing an excellent academic atmosphere.
I owe debts to Mathew Englander, Adam Leibovich and Daniel Williams
for critical reading of early drafts; Steve Frautschi, David
Goodstein, Andrew Lange, Bob McKeown and Harvey Newman for defining,
by example, excellent pedagogy; and mentors Michel Baranger, Roger
Blandford, Gerry Neugebauer and Scott Tremaine for shaping my picture
of physics in general.

\vspace{1ex}
\noindent
David W. Hogg \\
{\em Princeton, New Jersey\/} \\
{\em November 1997\/}
