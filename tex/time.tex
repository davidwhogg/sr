\chapter{Time dilation and length contraction}
\label{chap:time}

This Chapter is intended to demonstrate the simplicity of special
relativity.  With one basic thought experiment the two most important
effects predicted by the theory are derived: time dilation and length
contraction.  For the beginning student of relativity, this is the
most important chapter.

It is emphasized that the predicted effects are real, not just
apparent.

Before starting, recall Einstein's (1905) principle of relativity
(hereafter ``the'' principle of relativity): there is no preferred
reference frame; no entirely on-board experiment can tell a sailor the
speed of her or his boat.  Its first consequence is that the speed of
light is the same in all frames.


\section{Time dilation}
\label{sec:timedilation}

Consider two observers, Deepto (D) and Erika (E), moving relative to
one another in spaceships.  D measures E's speed to be $u$ with
respect to D's rest frame.  By symmetry, E must also measure D's speed
to be $u$ with respect to E's rest frame.  If this is not obvious to
you, notice that there is no absolute difference between D and E.  If
they did not measure the same speed, which one of them would measure a
higher speed?  In order for one to measure a higher speed, one of them
would have to be in a special or ``preferred'' frame; the principle of
relativity precludes this.

Now imagine that D and E each carry a clock of a certain very strange
type.  These ``light-clocks''\index{light-clock} consist of an
evacuated glass tube containing a lightbulb, a mirror, a photodetector
and some electrical equipment.  The photodetector is right next to the
lightbulb (but separated by a light-blocking shield) and the mirror is
0.5 m from the lightbulb (see Figure~\ref{fig:lightclock}).  When the
clock is started, the lightbulb flashes, light bounces off the mirror
and back into the photodetector.  When the photodetector registers the
light, it immediately signals the lightbulb to flash again.  Every
time the photodetector registers a light pulse, it flashes the bulb
again.
\hoggfigure{Schematic of a light-clock}{\horfigure{lightclock.eps}}%
{The schematic layout of a light-clock.  The round-trip distance
(lightbulb to mirror to photodetector) for the light is 1 m.}{fig:lightclock}

The round-trip distance for the light inside the light-clock is 1~m
and the speed of light $c$ is roughly $3\times 10^8~\ms$, so the
round-trip time for the light is roughly $3.3\times 10^{-9}~{\rm s}$.
The clock ticks off time in $3.3~{\rm ns}$ (nanosecond) intervals.
The speed of light is the same for all observers, so $c$ can be seen
as a conversion factor between time and distance.  Under this
interpretation, the clock ticks off time in meters!\footnote{Now that
the meter is defined in terms of the second, this is in fact the
interpretation of the speed of light that the International Standards
Organization accepts.  The speed of light is {\em defined\/} to be
$2.99792458\times 10^8~\ms$.}

% Need a reference for the definition thing.

Imagine that D holds his light-clock so that the light is bouncing
back and forth at right angles to his direction of motion with respect
to E.  D observes the light flashes in his clock to make 1 m round
trips in $\Delta t=3.3~{\rm ns}$ intervals.  What does E observe?
Recall that D is moving at speed $u$ with respect to E, so in E's rest
frame the light in D's clock is not really making round trips.  As it
travels down the tube and back, D is advancing in the perpendicular
direction; the light takes a zig-zag path which is longer than the
straight back-and-forth path (see Figure~\ref{fig:trajectories}).  By
the principle of relativity, E and D must observe the same speed of
light, so we are forced to conclude that E will measure\footnote{What
is meant by ``measure'' here is explained in the next Section---Erika
is a very good scientist!} longer time intervals $\Delta t'$ between
the flashes in D's clock than D will.  (In this chapter, all
quantities that E measures will be primed and all that D measures will
be unprimed.)  What is the difference between $\Delta t$ and $\Delta
t'$?
\hoggfigure{Trajectories of light seen by D and E}{\horfigure{trajectories.eps}}%
{The trajectory of the
light in D's light-clock, as observed by (a)~D and (b)~E.  Note that
the light follows a longer path in E's frame, so E measures a longer
time interval $\Delta t'$.}{fig:trajectories}

In E's rest frame, in time $\Delta t'$, D advances a distance $\Delta
x'=u\,\Delta t'$ and the light in D's clock must go a total distance
$\Delta\ell'=c\,\Delta t'$.  By the Pythagorean theorem
$(\Delta\ell')^2=(\Delta x')^2+(\Delta y)^2$, where $\Delta y$ is the
total round-trip length of the clock ($1~{\rm m}$ in this case) in its
rest frame and for now it has been assumed that $\Delta y=\Delta y'$
(this will be shown in Section~\ref{sec:lengthcontraction}).  Since
$\Delta y=\Delta\ell =c\,\Delta t$, we find
\begin{equation}
\Delta t'=\frac{\Delta t}{\sqrt{1-\frac{u^2}{c^2}}}
\end{equation}
The time intervals between flashes of D's clock are longer as
measured by E than as measured by D.  This effect is called {\em time
dilation.\/}\index{time dilation} Moving clocks go slow.

It is customary to define the dimensionless speed $\beta$ and the {\em
Lorentz factor\/} $\gamma$ by
\begin{equation}
\beta\equiv\frac{u}{c}
\end{equation}
\begin{equation}
\gamma\equiv\frac{1}{\sqrt{1-\beta^2}}
\end{equation}
Because (as we shall see later) nothing travels faster than the speed
of light, $u$ is always less than $c$, so $0\leq\beta <1$, and
$\gamma\geq 1$.  Using these new symbols, $\Delta t'=\gamma\,\Delta
t$.

Above we found that ``moving clocks go slow,'' but one might object
that we have shown only that these strange light-clocks go slow.
However, we can show that {\em all\/} clocks are subject to the same
time dilation.  Suppose that in addition to his light-clock, D also
has a wristwatch that ticks every $3.3~{\rm ns}$, and suppose
(incorrectly) that this watch is not subject to time dilation; i.e.,
suppose that E observes the watch to tick with intervals of $3.3~{\rm
ns}$ no matter what D's speed.  When D is not moving with respect to E
the wristwatch and light-clock tick at the same rate, but when D is
moving at high speed, they tick at different rates because, by
supposition, one is time-dilated and the other is not.  D could use
the relative tick rates of the watch and clock to determine his speed,
and thereby violate the principle of relativity.  It is left to the
ambitious reader to prove that it is not possible for D to observe
both timepieces to tick at the same rate while E observes them to tick
at different rates.

The reader might object that we have already violated relativity: if D
and E are in symmetric situations, how come E measures longer time
intervals?  We must be careful.  E measures longer time intervals for
D's clock than D does.  By relativity, it must be that D {\em also\/}
measures longer time intervals for E's clock than E does.  Indeed this
is true; after all, all of the above arguments are equally applicable
if we swap D and E.  This is the fundamentally counterintuitive aspect
of relativity.  How it can be that both observers measure slower rates
on the other's clock?  The fact is, there is no contradiction, as long
as we are willing to give up on a concept of absolute time,
agreed-upon by all observers.  The next two Chapters explore this and
attempt to help develop a new intuition.

\begin{problem}
Your wristwatch ticks once per second.  What is the time interval
between ticks when your wristwatch is hurled past you at half the
speed of light?
\end{problem}

\begin{problem}
How fast does a clock have to move to be ticking at one tenth of its
rest tick rate?  One one-hundredth?  One one-thousandth?  Express your
answers in terms of the difference $1-\beta$, where of course
$\beta\equiv v/c$.
\end{problem}

\begin{problem}
Consider the limit in which $\gamma\gg 1$, so its inverse $1/\gamma$
is a small number.  Derive an approximation for $\beta$ of the form
$\beta\approx 1-\epsilon$ which is correct to second order in
$1/\gamma$.
\end{problem}

\begin{problem}
Consider the low-speed limit, in which $\beta\ll 1$.  Derive an
expression for $\gamma$ of the form $\gamma\approx 1+\epsilon$ which
is correct to second order in $\beta$.
\end{problem}

\begin{problem}
Prove (by thought experiment) that it is not possible for D to observe
both his light-clock and his wristwatch to tick at the same rate while
E observes them to tick at different rates.  ({\em Hint:\/} Imagine
that both of D's clocks punch a ticker tape and the experimentalists
compare the tapes after the experiment is over.)
\end{problem}


\section{Observing time dilation}
\label{sec:observing}

In the previous section, as in the rest of these notes, it is
important to distinguish between what an ideally knowledgeable
observer\index{observer, definition of} {\em observes\/} and what an
ordinary person {\em sees.\/} As much as possible, the term ``to
observe'' will be used to mean ``to measure a real effect with a
correct experimental technique,'' while ``to see'' will be reserved
for apparent effects, or phenomena which relate to the fact that we
look from a particular viewpoint with a particular pair of eyes.  This
means that we won't talk about what is ``seen'' in detail until
Chapter~\ref{chap:astronomy}.

Though E {\em observes\/} D's clock to run slow, what she {\em sees\/}
can be quite different.  The time intervals between the flashes of D's
clock that she sees depends on the time dilation {\em and\/} the
changing path lengths that the light traverses in getting to E.  The
path lengths are changing because D is moving with respect to E (see
Figure~\ref{fig:observing}).  In order to correctly measure the rate
of D's clock, E must subtract the light-travel time of each pulse
(which she can compute by comparing the direction from which the light
comes with the trajectory that was agreed upon in advance).  It is
only when she subtracts these time delays that she measures the time
between ticks correctly, and when she does this, she will find that
the time between ticks is indeed $\Delta t'$, the dilated time.
\hoggfigure{Observing time delay}{\horfigure{observing.eps}}%
{Observing the time delay.  Because D is
moving with respect to E, the flashes ($F_1$ through $F_4$) from his
clock travel along paths ($S_1$ through $S_4$) of different lengths in
getting to E.  Hence different flashes take different times to get to
E.  E must correct for this before making any statements about time
dilation.  It is after the correction is made that E observes the
predicted time dilation.}{fig:observing}

\begin{problem}
Consider a clock, which when at rest produces a flash of light every
second, moving away from you at $(4/5)c$.  (a)~How frequently does it
flash when it is moving at $(4/5)c$?  (b)~By how much does distance
between you and the clock increase between flashes?  (c)~How much
longer does it take each flash to get to your eye than the previous
one?  (d)~What, therefore, is the interval between the flashes you
see?
\end{problem}

You will find that the time interval between the flashes you see is
much longer than merely what time-dilation predicts, because
successive flashes come from further and further away.  This effect is
known as the Doppler shift and is covered in much more detail in
Chapter~\ref{chap:astronomy}


\section{Length contraction}
\label{sec:lengthcontraction}

Imagine that E observes D's clock to tick 100 times during a journey
from planet A to planet B, two planets at rest in E's rest frame.

D must also observe 100 ticks during this same journey.  After all, if
we imagine that the clock punches a time card each time it ticks and D
inserts the time card at point A and removes it at point B, it must
have been punched a definite number of times when it is removed.  D
and E must agree on this number, because, for example, they can meet
later and examine the card.

In addition to agreeing on the number of ticks, D and E also agree on
their relative speed.  (They must, because there is total symmetry
between them: if one measured a larger speed, which one could it be?)
However, they do {\em not\/} agree on the rate at which D's clock
ticks.  While E measures the distance between A and B to be
$\ell'=100\, u\,\Delta t'$, D measures it to be $\ell=100\, u\,\Delta
t=\ell'/\gamma$.  Since $\gamma >1$, D measures a {\em shorter\/}
distance than E.  D is moving relative to the planets A and B, while E
is stationary.  Planets A and B can be thought of as being at the ends
of a ruler stick which E is holding, a ruler stick which is moving
with respect to D.  We conclude that moving ruler sticks are
shortened; this effect is {\em length contraction,\/} or sometimes
{\em Lorentz contraction.\/}

It is simple to show that length contraction acts only parallel to the
direction of motion.  Imagine that both E and D are carrying identical
pipes, aligned with the direction of their relative motion (see
Figure~\ref{fig:pipes}).
\hoggfigure{E and D carrying pipes}{\horfigure{pipes.eps}}%
{E and D carrying pipes to prove that
there can be no length changes perpendicular to the direction of
motion.}{fig:pipes}
Let us assume (incorrectly) that the large relative velocity causes
the diameter of E's pipe to contract in D's frame.  If this happens,
D's pipe becomes larger than E's pipe, so E's pipe ``fits inside'' D's
pipe.  But E and D are interchangeable, so D's pipe contracts in E's
frame and D's pipe fits inside E's.  Clearly it cannot be that both
D's fits inside E's and E's fits inside D's, so there is a
contradiction; there can be no length changes perpendicular to the
direction of relative motion.

Note that because there are no length changes perpendicular to the
direction of motion, we cannot explain away time dilation and length
contraction with length changes in the light-clock perpendicular to
the direction of motion.

\begin{problem}
How fast do you have to throw a meter stick to make it one-third its
rest length?
\end{problem}

\begin{problem}
Two spaceships, each measuring $100~{\rm m}$ in its own rest frame,
pass by each other traveling in opposite directions.  Instruments on
board spaceship A determine that the front of spaceship B requires
$5\times10^{-6}~{\rm s}$ to traverse the full length of A.  (a)~What
is the relative velocity $v$ of the two spaceships?  (b)~How much time
elapses on a clock on spaceship B as it traverses the full length of
A?  (From French 1966.)
\end{problem}

\begin{problem}
That there can be no length contraction perpendicular to the direction
of motion is often demonstrated with the example of a train and its
track; i.e., if there were length changes perpendicular the train
would no longer fit on the track.  Make this argument, and in
particular, explain why the train must fit on the track no matter how
fast it is going.
\end{problem}


\section{Magnitude of the effects}
\label{sec:mageff}

As these example problems show, the effects of time dilation and
length contraction are extremely small in everyday life, but large
for high-energy particles and any practical means of interstellar
travel.

\begin{problem}
In the rest frame of the Earth, the distance $\ell$ between New York
and Los Angeles is roughly $4000$~km.  By how much is the distance
shortened when observed from a jetliner flying between the cities?
From the Space Shuttle?  From a cosmic ray proton traveling at $0.9c$?
\end{problem}

In the rest frame, the distance is $\ell$; to an observer traveling at
speed $u$ along the line joining the cities, it is
$\ell'=\ell/\gamma$.  The difference is
\begin{equation}
\ell-\ell' = \left(1-\frac{1}{\gamma}\right)\,\ell
     = \left(1-\sqrt{1-\beta^2}\right)\,\ell
\end{equation}
For $\epsilon$ much smaller than unity, $(1+\epsilon)^n\approx
1+n\,\epsilon$, so for speeds $u\ll c$ or $\beta\ll 1$, we have
\begin{equation}
\ell-\ell'\approx \frac{1}{2}\,\beta^2\,\ell
\end{equation}

A jetliner takes about $6~{\rm h}$ to travel from New York to Los
Angeles, so its speed is roughly $u=4000/6~\kmh$ or $\beta=6\times
10^{-7}$.  Since $\beta\ll 1$, we have that $\ell-\ell'\approx 8\times
10^{-7}~{\rm m}$, or $0.8$~microns!  The Space Shuttle takes about
$1.5~{\rm h}$ to orbit the earth, on an orbit with radius roughly
$6500~{\rm km}$, so $\beta=2.5\times 10^{-5}$.  Here $\ell-\ell'\approx
1.3~{\rm mm}$.

As for the cosmic ray proton, $\beta=0.9$, so it is no longer true
that $\beta\ll 1$; we gain nothing by using the approximation.  We
find $\gamma=2.3$ and so $\ell-\ell'=2300~{\rm km}$.

\begin{problem}
At rest in the laboratory, muons have a mean life $T$ of $2.2\times
10^{-6}~{\rm s}$ or $2.2~{\rm \mu s}$, or in other words, the average
time a muon exists from production (in a collision, say) to decay
(into an electron and neutrinos) is $2.2~{\rm \mu s}$ (Particle Data
Group, 1994).  If, as experimentalists, we need a sample of muons to
have a longer mean life of $T'=11~{\rm \mu s}$, to what speed $u$ must
we accelerate them?  What distance $\ell$, on average, does one of
these high-speed muons travel before decaying?
\end{problem}

We want the muons to age at $1/5$ their usual rate, so we want time
dilation by a factor $\gamma=5$.  Inverting the formula for $\gamma$
we find
\begin{equation}
\beta=\sqrt{1-\frac{1}{\gamma^2}}
\end{equation}
or in this case $\beta=24/25$.  This makes $u=24c/25$ and
$\ell=u\,T=630~{\rm m}$.

\begin{problem}
Alpha Centauri is a distance of $\ell=4.34$~light years (one light
year is the distance light travels in one year) from the Earth.  At
what speed $u$ must a 25-year-old astronaut travel there and back if
he or she is to return before reaching age 45?  By how much will the
astronaut's siblings age over the same time?
\end{problem}

This is the famous ``twin paradox,'' which we will cover in gory
detail in Section~\ref{sec:twins}.  For now, let us be simplistic and
answer the questions without thinking.

We want the elapsed time $T'$ in the astronaut's frame to be 20 years
as he or she goes a distance $2\ell'$, the distance from the Earth to
Alpha Centauri and back in the astronaut's frame.  The time and
distance are related by $T'=2\ell'/u=2\ell/(\gamma u)$.  So we need
$\gamma u=2\ell/T'$.  Dividing by $c$, squaring and expanding we need
\begin{equation}
\frac{\beta^2}{1 - \beta^2} = \left(\frac{2\,\ell}{c\,T'}\right)^2
 = (0.434)^2
\end{equation}
This is a linear equation for $\beta^2$; we find $\beta=0.398$.  So
the astronaut must travel at $u=0.398c$, and from the point of view of
the siblings, the trip takes $T=2\ell /u=21.8~{\rm yr}$.


\section{Experimental confirmation}
\label{sec:experimental}

As we have seen in the previous section, the effects of time dilation
and length contraction are not very big in our everyday experience.
However, these predictions of special relativity {\em have\/} been
confirmed experimentally.  Time dilation is generally easier to
confirm directly because Nature provides us with an abundance of
moving clocks, and because in such experiments, it is generally more
straightforward to design procedures in which the delays from light
travel time (discussed in Section~\ref{sec:observing}) are not
important. Of course in addition to experiments like the one discussed
in this section, both time dilation and length contraction are
confirmed indirectly countless times every day in high energy physics
experiments around the world.

The first direct confirmation of time dilation was obtained by Bruno
Rossi\index{Rossi, B.} and David Hall,\index{Hall, D. B.} studying the
decay of muons (in those days called ``mesotrons'' or ``mu mesons'')
as they descend through the Earth's at\-mo\-sphere.\footnote{The
information in this section comes from Rossi \& Hall (1941), their
extremely readable, original paper.}  Muons are elementary
particles\footnote{For those who care, muons are {\em leptons,\/} most
analogous to electrons, with the same charge but considerably more
mass.  They are unstable and typically decay into electrons and
neutrinos.} produced at high altitude when cosmic rays (fast-moving
protons and other atomic nuclei) collide with atoms in the Earth's
atmosphere.  When produced more or less at rest in the laboratory,
each muon has a mean lifetime of $\tau_0=2.5\times 10^{-6}$ seconds
before it disintegrates.  Indeed, if one has $N_0$ muons at time zero
and then looks at a later time $t$, the number of muons will have
dropped to $N(t)=N_0\,e^{-t/\tau_0}$.  If there were no such thing as
time dilation, the mean distance a muon moving at high speed $v=\beta
c$ could travel before disintegrating would be $L=v\tau_0$.  Similarly
if at position zero one has $N_0$ muons moving at speed $v$ down a
tube, at a position $x$ further down the tube there would be only
$N(x)=N_0\,e^{-x/L}$.  As the speed of the muons approaches $c$, the
mean range would approach $c\tau_0$, or $750$~m.  Since the muons are
created at high altitude, very few of them could reach the ground.

However, we expect that time dilation {\em does\/} occur, and so the
mean life $\tau$ and range $L$ of the moving muons will be increased
by the Lorentz factor $\gamma\equiv(1-\beta^2)^{-1/2}$ to
$\tau=\gamma\,\tau_0$ and $L=\gamma\,v\,\tau_0$.  Although all the
muons will be moving at speeds close to $c$ ($\beta$ nearly 1), they
will have different particular values of $\gamma$ and therefore decay
with different mean ranges.  Bruno \& Rossi measure the fluxes (number
of muons falling on a detector of a certain area per minute) of muons
of two different kinetic energies at observing stations in Denver and
Echo Lake, Colorado, separated in altitude by $\Delta h=1624$~m
(Denver below Echo Lake).  The higher-energy muons in their experiment
have Lorentz factor $\gamma_1\approx 18.8$ (speed $v_1\approx
0.9986c$) while the lower-energy muons have $\gamma_2\approx 6.3$
($v_2\approx 0.987c$).  Because we expect the mean range $L$ of a muon
to be $L=\gamma\,v\,\tau_0$, we expect the ratio of ranges $L_1/L_2$
for the two populations of muons to be $(\gamma_1\, v_1)/(\gamma_2\,
v_2)\approx 3.0$.  The flux of higher-energy muons at Denver is lower
by a factor of $0.883\pm 0.005$ than it is at Echo Lake, meaning that
if they have mean range $L_1$, $e^{-\Delta h/L_1}=0.883$.  The flux of
lower-energy muons decreases by a factor of $0.698\pm 0.002$, so
$e^{-\Delta h/L_2}=0.698$.  Taking logarithms and ratios, we find that
$L_1/L_2=2.89$ as predicted.  The results do not make sense if the
time dilation factor (the Lorentz factor) is ignored.

%[Add a description of Lorentz's work on contraction of the electric
%field of moving charges?]

\begin{problem}
Consider a muon traveling straight down towards the surface of the
Earth at Lorentz factor $\gamma_1\approx 18.8$.  (a)~What is the
vertical distance between Denver and Echo Lake, according to the muon?
(b)~How long does it take the muon to traverse this distance,
according to the muon?  (c)~What is the muon's mean lifetime,
according to the muon?  (d)~Answer the above parts again but now for
a muon traveling at Lorentz factor $\gamma_2\approx 6.3$.
\end{problem}

\begin{problem}
Charged pions are produced in high-energy collisions between protons
and neutrons.  They decay in their own rest frame according to the law
\begin{equation}
N(t) = N_0\,2^{-t/T}
\end{equation}
where $T=2\times 10^{-8}~{\rm s}$ is the half-life.  A burst of pions
is produced at the target of an accelerator and it is observed that
two-thirds of them survive at a distance of $30~{\rm m}$ from the
target.  At what $\gamma$ value are the pions moving?  (From French
1966.)
\end{problem}

\begin{problem}
A beam of unstable $\rm K^+$ mesons, traveling at speed
$\beta=\sqrt{3}/2$, passes through two counters $9~{\rm m}$ apart.
The particles suffer a negligible loss of speed and energy in passing
through the counters but give electrical pulses that can be counted.
It is observed that $1000$ counts are recorded in the first counter
and $250$ in the second.  Assuming that this whole decrease is due to
decay of the particles in flight, what is their half-life as it would
be measured in their own rest frame?  (From French 1966.)
\end{problem}
