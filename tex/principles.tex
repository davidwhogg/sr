\chapter{Principles of relativity}
\label{chap:principles}

These notes are devoted to the consequences of Einstein's (1905)
principle of special relativity, which states that all the fundamental
laws of physics are the same for all uniformly moving
(non-accelerating) observers.  In particular, all of them measure
precisely the same value for the speed of light in vacuum, no matter
what their relative velocities.  Before Einstein wrote, several
principles of relativity had been proposed, but Einstein was the first
to state it clearly and hammer out all the counterintuitive
consequences.  In this Chapter the concept of a ``principle of
relativity'' is introduced, Einstein's is presented, and some of the
experimental evidence prompting it is discussed.


\section{What is a principle of relativity?}
\label{sec:whatis}

The first principle of relativity ever proposed is attributed to
Galileo, although he probably did not formulate it precisely.
Galileo's principle of relativity says that sailors on a uniformly
moving boat cannot, by performing on-board experiments, determine the
boat's speed.  They can determine the speed by looking at the relative
movement of the shore, by dragging something in the water, or by
measuring the strength of the wind, but there is no way they can
determine it without observing the world outside the boat.  A sailor
locked in a windowless room cannot even tell whether the ship is
sailing or docked\footnote{The sailor is not allowed to use some
characteristic rocking or creaking of the boat caused by its motion
through the water.  That is cheating and anyway it is possible to make
a boat which has no such property on a calm sea}.

This is a principle of relativity, because it states that there are no
observational consequences of absolute motion.  One can only measure
one's velocity {\em relative\/} to something else.

As physicists we are empiricists: we reject as meaningless any concept
which has no observable consequences, so we conclude that there is no
such thing as ``absolute motion.''  Objects have velocities only with
respect to one another.  Any statement of an object's speed must be
made with respect to something else.

Our language is misleading because we often give speeds with no
reference object.  For example, a police officer might say to you
``Excuse me, but do you realize that you were driving at 85 miles per
hour?''  The officer leaves out the phrase ``with respect to the
Earth,'' but it is there implicitly.  In other words, you cannot
contest a speeding ticket on the strength of Galileo's principle since
it is implicit in the law that the speed is to be measured with
respect to the road.

When Kepler first introduced a heliocentric model of the Solar System,
it was resisted on the grounds of common sense.  If the Earth is
orbiting the Sun, why can't we ``feel'' the motion?  Relativity
provides the answer: there are no local, observational consequences to
our motion.\footnote{Actually, there {\em are\/} some observational
consequences to the Earth's {\em rotation\/} (spin): for example,
Foucault's pendulum, the existence of hurricanes and other rotating
windstorms, and the preferred direction of rotation of draining water.
The point here is that there are no consequences to the Earth's linear
motion through space.}  Now that the Earth's motion is generally
accepted, it has become the best evidence we have for Galilean
relativity.  On a day-to-day basis we are not aware of the motion of
the Earth around the Sun, despite the fact that its orbital speed is a
whopping $30~\kms$ ($100,000~\kmh$).  We are also not aware of the
Sun's $220~\kms$ motion around the center of the Galaxy (e.g., Binney
\& Tremaine 1987, Chapter~1) or the roughly $600~\kms$ motion of the
local group of galaxies (which includes the Milky Way) relative to the
rest frame of the cosmic background radiation (e.g., Peebles 1993,
Section~6).  We have become aware of these motions only by observing
extraterrestrial references (in the above cases, the Sun, the Galaxy,
and the cosmic background radiation).  Our everyday experience is
consistent with a stationary Earth.

\begin{problem}
You are driving at a steady $100~\kmh$.  At noon you pass a parked
police car.  At twenty minutes past noon, the police car passes you,
travelling at $120~\kmh$.  (a)~How fast is the police car moving
relative to you?  (b)~When did the police car start driving, assuming
that it accelerated from rest to $120~\kmh$ instantaneously?  (c)~How
far away from you was the police car when it started?
\end{problem}

\begin{problem}
You are walking at $2~\ms$ down a straight road, which is aligned with
the $x$-axis.  At time $t=0$~s you sneeze.  At time $t=5$~s a dog
barks, and at the moment he barks he is $x=10$~m ahead of you in the
road.  At time $t=10$~s a car which is just then $15$~m behind you
($x=-15$~m) backfires.  (a)~Plot the positions $x$ and times $t$ of
the sneeze, bark and backfire, relative to you, on a two-dimensional
graph.  Label the points.  (b)~Plot positions $x'$ and times $t'$ of
the sneeze, bark and backfire, relative to an observer standing still,
at the position at which you sneezed.  Assume your watches are
synchronized.
\end{problem}

\begin{problem}
If you throw a superball at speed $v$ at a wall, it bounces back with
the same speed, in the opposite direction.  What happens if you throw
it at speed $v$ towards a wall which is travelling towards you at
speed $w$?  What is your answer in the limit in which $w$ is much larger
than $v$?
\end{problem}

\begin{problem}
You are trying to swim directly east across a river flowing south.
The river flows at $0.5~\ms$ and you can swim, in still water, at
$1~\ms$.  Clearly if you attempt to swim directly east you will drift
downstream relative to the bank of the river.  (a)~What angle
$\theta_a$ will your velocity vector relative to the bank make with
the easterly direction?  (b)~What will be your speed (magnitude of
velocity) $v_a$ relative to the bank?  (c)~To swim directly east
relative to the bank, you need to head upstream.  At what angle
$\theta_c$ do you need to head, again taking east to be the zero of
angle?  (d)~When you swim at this angle, what is your speed $v_c$
relative to the bank?
\end{problem}


\section{Einstein's principle of relativity}
\label{sec:einstein}

Einstein's principle of relativity says, roughly, that every physical
law and fundamental physical constant (including, in particular, the
speed of light in vacuum) is the same for all non-accelerating
observers.  This principle was motivated by electromagnetic theory and
in fact the field of special relativity was launched by a paper
entitled (in English translation) ``On the electrodynamics of moving
bodies'' (Einstein 1905).\footnote{This paper is extremely readable
and it is strongly reccomended that the student of relativity read it
during a course like this one.  It is available in English translation
in Lorentz et al.\ (1923).}  Einstein's principle is not different
from Galileo's except that it explicitly states that electromagnetic
experiments (such as measurement of the speed of light) will not tell
the sailor in the windowless room whether or not the boat is moving,
any more than fluid dynamical or gravitational experiments.  Since
Galileo was thinking of experiments involving bowls of soup and
cannonballs dropped from towers, Einstein's principle is effectively a
generalization of Galileo's.

The governing equations of electromagnetism, Maxwell's equations
(e.g., Purcell 1985), describe the interactions of magnets, electrical
charges and currents, as well as light, which is a disturbance in the
electromagnetic field.  The equations depend on the speed of light $c$
in vacuum; in other words, if the speed of light in vacuum was
different for two different observers, the two observers would be able
to tell this simply by performing experiments with magnets, charges
and currents.  Einstein guessed that a very strong principle of
relativity might hold, that is, that the properties of magnets,
charges and currents will be the same for all observers, no matter
what their relative velocities.  Hence the speed of light must be the
same for all observers.  Einstein's guess was fortified by some
experimental evidence available at the time, to be discussed below,
and his principle of relativity is now one of the most rigorously
tested facts in all of physics, confirmed directly and indirectly in
countless experiments.

The consequences of this principle are enormous.  In fact, these notes
are devoted to the strange predictions and counterintuitive results
that follow from it.  The most obvious and hardest to accept (though
it has been experimentally confirmed countless times now) is that the
following simple rule for velocity addition (the rule you must have
used to solve the Problems in the previous Section) is false:

Consider a sailor Alejandro (A) sailing past an observer Barbara (B)
at speed $u$.  If A throws a cantaloupe, in the same direction as he
is sailing past B, at speed $v'$ relative to himself, B will observe
the cantaloupe to travel at speed $v=v'+u$ relative to herself.  {\em
This rule for velocity addition is wrong.} Or imagine that A drops
the cantaloupe into the water and observes the waves traveling forward
from the splash.  If B is at rest with respect to the water and water
waves travel at speed $w$ relative to the water, B will obviously see
the waves travel forward from the splash at speed $w$.  On the other
hand A, who is moving forward at speed $u$ already, will see the waves
travel forward at lower speed $w'=w-u$.  {\em This rule for velocity
addition is also wrong!}

After all, instead of throwing a cantaloupe, A could have shined a
flashlight.  In this case, if we are Galileans (that is, if we believe
in the above rule for velocity addition), there are two possible
predictions for the speeds at which A and B observe the light to
travel from the flashlight.  If light is made up of particles which
are emitted from their source at speed $c$ relative to the source,
then A will observe the light to travel at speed $c$ relative to
himself, while B will observe it to travel at $c+u$ relative to
herself.  If, on the other hand, light is made up of waves that travel
at $c$ relative to some medium (analogous to the water for water
waves), then we would expect A to see the light travel at $c-u$ and B
to see it travel at $c$ (assuming B is at rest with the medium).
Things get more complicated if both A and B are moving relative to the
medium, but in almost every case we expect A and B to observe
different speeds of light if we believe our simple rule for velocity
addition.

Einstein's principle requires that A and B observe exactly the same
speed of light, so Einstein and the simple rules for velocity addition
cannot both be correct.  It turns out that Einstein {\em is\/} right
and the ``obvious'' rules for velocity addition are incorrect.  In
this, as in many things we will encounter, our initial intuition is
wrong.  We will try to build a new, correct intuition based on
Einstein's principle of relativity.

\begin{problem}
(For discussion.)  What assumptions does one naturally make which must
be wrong in order for A and B to measure the same speed of light in
the above example?  Consider how speeds are measured: with rulers and
clocks.
\end{problem}


\section{The Michelson-Morley experiment}
\label{sec:mme}

In the late nineteenth century, most physicists were convinced, contra
Newton (1730), that light is a wave and not a particle phenomenon.
They were convinced by interference experiments whose results can be
explained (classically) only in the context of wave optics.  The fact
that light is a wave implied, to the physicists of the nineteenth
century, that there must be a medium in which the waves
propagate---there must be something to ``wave''---and the speed of
light should be measured relative to this medium, called {\em the
aether.\/} (If all this is not obvious to you, you probably were not
brought up in the scientific atmosphere of the nineteenth century!)
The Earth orbits the Sun, so it cannot be at rest with respect to the
medium, at least not on every day of the year, and probably not on any
day.  The motion of the Earth through the aether can be measured with
a simple experiment that compares the speed of light in perpendicular
directions.  This is known as the Michelson-Morley experiment and its
surprising result was a crucial hint for Einstein and his
contemporaries in developing special relativity.\footnote{The
information in this section comes from Michelson \& Morley (1887) and
the history of the experiment by Shankland (1964).}

Imagine that the hypothesis of the aether is correct, that is, there
is a medium in the rest frame of which light travels at speed $c$, and
Einstein's principle of relativity does not hold.  Imagine further
that we are performing an experiment to measure the speed of light
$c_{\oplus}$ on the Earth, which is moving at velocity
$\tv{v}_{\oplus}$ (a vector with magnitude $v_{\oplus}$) with respect
to this medium.  If we measure the speed of light in the direction
parallel to the Earth's velocity $\tv{v}_{\oplus}$, we get
$c_{\oplus}=c-v_{\oplus}$ because the Earth is ``chasing'' the light.
If we measure the speed of light in the opposite
direction---antiparallel to the Earth's velocity---we get
$c_{\oplus}=c+v_{\oplus}$.  If we measure in the direction
perpendicular to the motion, we get
$c_{\oplus}=\sqrt{c^2-v_{\oplus}^2}$ because the speed of light is the
hypotenuse of a right triangle with sides of length $c_{\oplus}$ and
$v_{\oplus}$.\footnote{The demonstration of this is left as an
exercise.}  If the aether hypothesis is correct, these arguments show
that the motion of the Earth through the aether can be detected with a
laboratory experiment.

The Michelson-Morley experiment was designed to perform this
determination, by comparing directly the speed of light in
perpendicular directions.  Because it is very difficult to make a
direct measurement of the speed of light, the device was very cleverly
designed to make an accurate {\em relative\/} determination.  Light
entering the apparatus from a lamp is split into two at a
half-silvered mirror.  One half of the light bounces back and forth 14
times in one direction and the other half bounces back and forth 14
times in the perpendicular direction; the total distance travelled is
about $11~{\rm m}$ per beam.  The two beams are recombined and the
interference pattern is observed through a telescope at the output.
The whole apparatus is mounted on a stone platform which is floated on
mercury to stabilize it and allow it to be easily rotated.
Figure~\ref{fig:mm} shows the apparatus, and Figure~\ref{fig:m} shows
a simplified version.
\hoggfigure{The Michelson-Morley apparatus}{\horfigure{mm.eps}}{The Michelson-Morley
apparatus (from Michelson \& Morley 1887).  The light enters the
apparatus at $a$, is split by the beam splitter at $b$, bounces back
and forth between mirrors $d$ and $e$, $d_1$ and $e_1$, with mirror
$e_1$ adjustable to make both paths of equal length, the light is
recombined again at $b$ and observed through the telescope at $f$.  A
plate of glass $c$ compensates, in the direct beam, for the extra
light travel time of the reflected beam travelling through the beam
splitter an extra pair of times.  See Figure~\ref{fig:m} for a
simplified version.}{fig:mm}
\hoggfigure{The Michelson apparatus}{\horfigure{m.eps}}{The Michelson
apparatus (from Kleppner \& Kolenkow 1973), the predecessor to the
Michelson-Morley apparatus (Figure~\ref{fig:mm}).  The Michelson
apparatus shows more clearly the essential principle, although it is
less sensitive than the Michelson-Morley apparatus because the path
length is shorter.}{fig:m}

If the total length of travel of each beam is $\ell$ and one beam is
aligned parallel to $\tv{v}_{\oplus}$ and the other is aligned
perpendicular, the travel time in the parallel beam will be
\begin{equation}
t_{\|}=\frac{\ell}{2\,(c+v_{\oplus})}+\frac{\ell}{2\,(c-v_{\oplus})}
 =\frac{\ell\,c}{c^2-v_{\oplus}^2}
\end{equation}
because half the journey is made ``upstream'' and half ``downstream.''
In the perpendicular beam,
\begin{equation}
t_{\bot}=\frac{\ell}{\sqrt{c^2-v_{\oplus}^2}}
\end{equation}
because the whole journey is made at the perpendicular velocity.
Defining $\beta\equiv v_{\oplus}/c$ and pulling out common factors,
the difference in travel time, between parallel and perpendicular
beams, is
\begin{equation}
\Delta t=\frac{\ell}{c}\,
 \left[\frac{1}{1-\beta^2}-\frac{1}{\sqrt{1-\beta^2}}\right]
\end{equation}
For small $x$, $(1+x)^n\approx 1+n\,x$, so
\begin{equation}
\Delta t\approx\frac{\ell}{2c}\,\beta^2
\end{equation}
Since the apparatus will be rotated, the device will swing from having
one arm parallel to the motion of the Earth and the other
perpendicular to having the one perpendicular and the other parallel.
So as the device is rotated through a half turn, the time delay
between arms will change by twice the above $\Delta t$.

The lateral position of the interference fringes as measured in the
telescope is a function of the relative travel times of the light
beams along the two paths.  When the travel times are equal, the
central fringe lies exactly in the center of the telescope field.  If
the paths differ by one-half a period (one-half a wavelength in
distance units), the fringes shift by one-half of the fringe
separation, which is well resolved in the telescope.  As the apparatus
is rotated with respect to the Earth's motion through the aether, the
relative travel times of the light along the two paths was expected to
change by $0.4$ periods, because (in the aether model) the speed of
light depends on direction.  The expected shift of the interference
fringes was $0.4$ fringe spacings, but no shift at all was observed as
the experimenters rotated the apparatus.  Michelson and Morley were
therefore able to place upper limits on the speed of the Earth
$v_{\oplus}$ through the aether; the upper limits were much lower than
the expected speed simply due to the Earth's orbit around the Sun (let
alone the Sun's orbit around the Galaxy and the Galaxy's motion among
its neighboring galaxies).

Michelson and Morley concluded that something was wrong with the
standard aether theory; for instance, perhaps the Earth drags its
local aether along with it, so we are always immersed in locally
stationary aether.  In a famous paper, Lorentz (1904) proposed that
all moving bodies are contracted along the direction of their motion
by the amount exactly necessary for the Michelson-Morley result to be
null.  Both these ideas seemed too much like ``fine-tuning'' a so-far
unsubstatiated theory.

Einstein's explanation---that there is no aether and that the speed of
light is the same for all observers (and in all directions)---is the
explanation that won out eventually.\footnote{It was also Poincar\'e's
(1900) explanation.  Forshadowing Einstein, he said that the
Michelson-Morley experiment shows that absolute motion cannot be
detected to second order in $v/c$ and so perhaps it cannot be detected
to any order.  Poincar\'e is also allegedly the first person to have
named this proposal a ``principle of relativity.''}  The
Michelson-Morley experiment was an attempt by ``sailors'' (Michelson
and Morley) to determine the speed of their ``boat'' (the Earth)
without looking out the window or comparing to some other object, so
according to the principle of relativity, they were doomed to failure.

\begin{problem}
With perfect mirrors and light source, the Michelson-Morley apparatus
can be made more sensitive by making the path lengths longer.  Why is
a device with longer paths more sensitive?  The paths can be
lengthened by making the platform larger or adding more mirrors (see
Figure~\ref{fig:mm}).  In what ways would such modification also
degrade the performance of the device given imperfect mirrors and
light source (and other real-world concerns)?  Discuss the pros and
cons of such modifications.
\end{problem}

\begin{problem}
Show that under the hypothesis of a stationary aether, the speed of
light as observed from a platform moving at speed $v$, in the
direction perpendicular to the platform's motion, is $\sqrt{c^2-v^2}$.
For a greater challenge: what is the observed speed for an arbitrary
angle $\theta$ between the direction of motion and the direction in
which the speed of light is measured?  Your answer should reduce to
$c+v$ and $c-v$ for $\theta=0$ and $\pi$.
\end{problem}

It is worthy of note that when Michelson and Morley first designed
their experiment and predicted the fringe shift, they did not realize
that the speed of light perpendicular to the direction of motion of
the platform would be other than $c$.  This correction was pointed out
to them by Potier in 1881 (Michelson \& Morley, 1887).


\section{The ``specialness'' of special relativity}
\label{sec:approximations}

Why is this subject called ``special relativity,'' and not just
``relativity''?  It is because the predictions we make only strictly
hold in certain special situations.

Some of the thought experiments (and real experiments) described in
these notes take place on the surface of the Earth, and we will assume
that the gravitational field of the Earth (and all other planets and
stars) is negligible.\footnote{The fractional error that the Earth's
gravity introduces into the experiments we describe must depend only
on the acceleration due to gravity $g$, the parameters of each
experiment, and fundamental constants.  Fractional error is
dimensionless, and the most obvious fundamental constant to use is
$c$.  The ratio $g/c$ has dimensions of inverse time.  This suggests
that an experiment which has a characteristic time $\tau$ or length
$\ell$ will not agree with the predictions of special relativity to
better than a fractional error of about $\tau\, g/c$ or $\ell\, g/c^2$
if it is performed on the surface of the Earth.}  The laws of special
relativity strictly hold only in a ``freely falling'' reference frame
in which the observers experience no gravity.  The laws strictly hold
when we are falling towards the Earth (as in a broken elevator; e.g.,
Frautschi et al., 1986, ch.~9) or orbiting around the Earth (as in the
Space Shuttle; ibid.), but not when we are standing on it.

Does the gravitational field of the Sun affect our results?  No,
because we are orbiting the Sun.  The Earth is in a type of ``free
fall'' around the Sun.  Does the rotation of the Earth affect our
results?  Yes, because the centrifugal force that is felt at the
equator is equivalent to an outward gravitational force.  However,
this effect is much smaller than the Earth's gravity, so it is even
more negligible.

In addition, we are going to assume that all light signals are
travelling in vacuum.  The speed of light in air is actually a bit
less than the speed of light in vacuum.  We will neglect this
difference.  The ``$c$'' that comes into the general equations that we
will derive is the speed of light in vacuum, no matter what the speed
at which light is actually travelling in the local medium.  Everything
is simpler if we just treat all our experiments as if they are
occurring in vacuum.

\begin{problem}
(Library excercise.)  How much slower (or faster) is the speed of
light in air, relative to vacuum?  How do you think the speed will
depend on temperature and pressure?  How much slower (or faster) is
the speed of light in glass and water, relative to vacuum.  Give your
references.
\end{problem}

\begin{problem}
You shine a flashlight from one end zone of a football field to a
friend standing in the other end zone.  Because of the Earth's
gravity, the beam of light will be pulled downwards as it travels
across the field.  Estimate, any way you can, the distance the light
will ``drop'' vertically as it travels across the field.  What
deflection angle does this correspond to, in arcseconds?
\end{problem}

Don't worry about getting a precise answer, just estimate the order of
magnitude of the effect.
